\begin{flushleft}
    \begin{thebibliography}{99}

    \bibitem{b1}
    Tanaka, T.; Hakuta, M.; Iwata, N.; Ohminami, M. (1995).
    [IEEE Comput. Soc. Press 12th TRON Project International Symposium - Tokyo, Japan (28 Nov.-2 Dec. 1995)]
    Proceedings of the 12th TRON Project International Symposium - Approaches to making software porting more productive. , (), 73–85.

    \bibitem{b2}
    Mitsuari Hakuta; Masato Ohminami (1997).
    A study of software portability evaluation. , 38(2), 145–154.

    \bibitem{b3}
    L. Fernando Capretz,
    "Bringing the Human Factor to Software Engineering"
    in IEEE Software, vol. 31, no. 02, pp. 104-104, 2014.

    \bibitem{b4}
    A. Kanai, T. Furuyama and M. Takahashi,
    "A cost model for software conversion based on program characteristics and a converter effect,"
    [1992] Proceedings. The Sixteenth Annual International Computer Software and Applications Conference, 1992, pp. 63-68

    \bibitem{b5}
    Mooney, J.D., 2004.
    Developing portable software.
    In Information Technology (pp. 55-84). Springer, Boston, MA.

    \bibitem{b6}
    Ejiogu, L.O., 1985.
    A simple measure of software complexity.
    ACM SIGPLAN Notices, 20(3), pp.16-31.

    \bibitem{b7}
    Mooney, J. D. (1990).
    Strategies for supporting application portability.
    Computer, 23(11), 59-70.

    \bibitem{b8}
    Cho, D. and Bae, D., 2011, June.
    Case Study on Installing a Porting Process for Embedded Operating System in a Small Team.
    In 2011 Fifth International Conference on Secure Software Integration and Reliability Improvement-Companion (pp. 19-25). IEEE.

    \bibitem{b9}
    Kanai, A., Furuyama, T. and Takahashi, M., 1992, January.
    A cost model for software conversion based on program characteristics and a converter effect.
    In 1992 Proceedings. The Sixteenth Annual International Computer Software and Applications Conference (pp. 63-64). IEEE Computer Society.

    \bibitem{b10}
    Porquet, J., 2015,
    Porting Linux to a new processor architecture,
    LWN,
    viewed 10 April 2022,
    <https://lwn.net/Articles/654783/>

    \bibitem{b11}
    Bodenstab, D. E., Houghton, T. F., Kelleman, K. A., Ronkin, G., \& Schan, E. P. (1984).
    The UNIX system: UNIX operating system porting experiences.
    AT\&T Bell Laboratories Technical Journal, 63(8), 1769-1790.

    \bibitem{b12}
    2015,
    Cross-Porting Software,
    osdev.org,
    viewed 10 April 2022,
    <https://wiki.osdev.org/Cross-Porting\_Software>

    \bibitem{b13}
    Jolitz, W. F., \& Jolitz, L. G. (1990).
    Porting UNIX to the 386: A practical approach.
    Dr. Dobb's Journal, 16(1), 16-46.

    \bibitem{b14}
    Frakes, W. B., \& Fox, C. J. (1995).
    Sixteen questions about software reuse.
    Communications of the ACM, 38(6), 75-ff.

    \bibitem{b15}
    Tanenbaum, A. S., Klint, P., \& Bohm, W. (1978).
    Guidelines for software portability.
    Software: Practice and Experience, 8(6), 681-698.

    \bibitem{b16}
    Johnson, S. C., \& Ritchie, D. M. (1978).
    UNIX time-sharing system: Portability of C programs and the UNIX system.
    The Bell System Technical Journal, 57(6), 2021-2048.

    \bibitem{b17}
    Xia, X., Bao, L., Lo, D., Xing, Z., Hassan, A.E. and Li, S., 2017.
    Measuring program comprehension: A large-scale field study with professionals.
    IEEE Transactions on Software Engineering, 44(10), pp.951-976.

    \bibitem{b18}
    Morgan, Malcolm J. (1994).
    Controlling Software Development Costs.
    Industrial Management \& Data Systems, 94(1), 13–18. doi:10.1108/02635579410053352

    \bibitem{b19}
    Boehm, B., Abts, C. and Chulani, S., 2000.
    Software development cost estimation approaches—A survey.
    Annals of software engineering, 10(1), pp.177-205.

    \bibitem{b20}
    Walli, S.R., 1995.
    The POSIX family of standards. StandardView, 3(1), pp.11-17.

    \bibitem{b21}
    Spencer, H. and Collyer, G., 1992.
    \# ifdef considered harmful, or portability experience with C News.
    In USENIX Summer 1992 Technical Conference (USENIX Summer 1992 Technical Conference).

    \end{thebibliography}
\end{flushleft}
