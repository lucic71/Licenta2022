\section{Introduction}

Porting a software system and evaluating the porting costs is a hard problem in
systems programming.~\cite{b1,b2,b4,b5,b9,b10,b11,b12,b13,b14,b15,b16} The
reasons for porting software systems are various: the developers want to enhance
the performance, the hardware environment starts to get deprecated or the system
wants to take advantage of features unavailable in the current environment.

This paper highlights the porting issues associated with porting the IxOS
testing infrastructure, regarding both the hardware and software components, to
ARM off-the-shelf boards, Raspberry Pi 4 in our case. In the \textit{Background}
section we present different terms associated with porting: what porting is, a
porting model, porting costs and factors. In the \textit{Porting IxOS on ARM
boards} we divide our work in multiple steps. For each step we present a
description, the targets and milestones, and the final results.  After the
porting process is discussed, we highlight the \textit{Porting Costs Evaluation}
associated with our work, including an analysis of the factors that influenced
the presented costs.  Finally, in \textit{Discussions on Porting Costs} we
conduct a discussion about the porting process and the porting costs where we
investigate the porting difficulties, observations about the porting tasks and
ways to improve the porting process. 
