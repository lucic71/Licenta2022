\section{Introduction}

Porting a software system and evaluating the porting costs is a hard problem in
systems programming.~\cite{b1,b2,b4,b5,b9,b10,b11,b12,b13,b14,b15,b16} The
reasons for porting software systems are various: the developers want to enhance
the performance, the hardware environment starts to get deprecated or the system
wants to take advantage of features unavailable in the current environment.

Software development costs and invested resources for understanding, maintaining
and developing new features for a system are big~\cite{b17,b18,b19}. We would
like to preserve these investments when the need for a new environment arises.
To achieve this goal people designed programming languages and compilers that
increased portability, operating systems that could run on multiple hardware
platforms~\cite{b16} and various standards that allowed developers to talk to
the computer using well defined interfaces~\cite{b20}.  However the porting
process is a non-trivial endeavor up to this day, it is error prone and time
consuming \todo{this sounds weird} without a proper understanding of the system and of the tasks involved
in the process. The happens because the initial architecture contains implicit
assumptions about the environment as: endianess, non-standard compiler behavior
or custom changes made in the operating system structure; or it may happen
because the developers make use of non-portable constructs as ifdefs \cite{b21}
or machine dependant code. Moreover these parts of the design and environment
are recorded in human-readable documents that are hard to preserve over time or
the information is not recorded at all. Thus the developer is forced to rely on
its own experience and skills or on the wise use of discussions to solve the
porting problems that might arise.

As with other software development processes, porting has its own costs
associated. It is important to evaluate them and understand their implications
in the project so that the developer can optimize the process of porting in the
future and call attention to the weaknesses and strengths of the process. In
this work we describe the experience of porting the IxOS testing infrastructure,
used by Ixia for high performance network testing, on ARM off-the-shelf boards.
We do this for two reasons. Firstly, we are interested in exploring Ixia testing
infrastructure on newer environments with the hope that we will improve the
performance of the system. Secondly, we are interested in evaluating porting
costs related to the project for the reasons listed in the beginning of this
paragraph. 

We make the following academic contributions in this work: we review the porting models
described in~\cite{b1,b2,b9} and propose a more general model that can be
applied to modern software porting, we review the porting costs factors
described in~\cite{b2} and analyze their relevance in our porting work and we
provide guidelines and discussions on the topic of improving software porting
based on the experience of porting IxOS infrastructure. On the technical side of
contributions, we separated the relevant parts for our porting from the
infrastructure and ported them on ARM architecture. By doing this we allow the
the developing of further testing tools and systems on ARM boards by Ixia.

\todo{I don't like how this sentece is formulated} This paper highlights the porting issues associated with porting the IxOS
testing infrastructure, regarding both the hardware and software components, to
ARM off-the-shelf boards, Raspberry Pi 4 in our case. \todo{Use references to
sections} In the \textit{Background}
section we present different terms associated with porting: what porting is, a
porting model, porting costs and factors. In the \textit{Porting IxOS on ARM
boards} we divide our work in multiple steps. For each step we present a
description, the targets and milestones, and the final results.  After the
porting process is discussed, we highlight the \textit{Porting Costs Evaluation}
associated with our work, including an analysis of the factors that influenced
the presented costs.  \todo{This sentence must be more accurate} Finally, in \textit{Discussions on Porting Costs} we
conduct a discussion about the porting process and the porting costs where we
investigate porting difficulties, observations about the porting tasks and
ways to improve the porting process. 
