\section{Conclusions and Further Work}

\subsection{Conclusions}
We succeeded in porting the IxOS infrastructure on ARM boards. We separated the
relevant components for our porting (i.e., InterfaceManager and IxStack) from
IxOS so that they run on every ARM-based Linux distribution. At the moment of
writing the project is in the proof-of-concept stage. If there is interest for
continuing the project or integrating it in other projects inside the company,
we provided the necessary environment for deploying it.

We have extracted the porting costs for porting IxOS infrastructure on ARM
boards. Thus we understood what the weaknesses and the strenghts of our project
were. The lack of understanding of the project structure and project use cases
proved itself to be an important factor during the process of porting. Because
of this reason we had to spend additional time on testing the system and
understanding its components that could have been used on solving problems and
inconsistencies. A strength of our project was the fact that the program for
porting had a high degree of portability. This helped us to shrink the volume of
inconsistencies between the source environment and the target environment.

We succeeded in creating a better model for software porting starting from the
model presented in~\cite{b1,b2} and from our project specific needs. We
contributed to this model by making it more generic and allowing other software
porting projects to easily map their needs on this model.

We have provided a discussion on the open problems in the area of software
porting including the non-linearity of the porting project, what implications
does this have in the extraction of porting costs and the need for a dependency
graph between the porting tasks so that progress tracking in software porting
projects could be done more efficiently. Finally, we compared our porting costs
with the costs of Tanaka et al.~\cite{b1} to understand better the differences
and patterns between our processes. 

\subsection{Further Work}

We plan to test the performance of the system so that we can achieve the other
goal we proposed in the beginning of this paper (i.e., explore Ixia testing
infrastructure on other architectures so that we can achieve a better
performance of the system). To achieve this goal we will compare our solution in
the target environment with the same solution in the source environment. We
expect to see better results in the new environment for some network testing
suites than in the old enviroment. Furthermore, it would be interesting to see
how our solution compares to other open-source solutions as {\color{red} TBD}.

To complete the analysis of the factors that affected the porting costs we plan
to analyze the characteristics of the program to be ported. We want to analyze
the program size and contents and the content of the changes needed for porting.
By doing this we hope to find a direct correspondence between the program to be
ported and specific porting subtasks (e.g., \textit{Solving inconsistencies
between source and target environment}).

Finally, we want to analyze the porting improvements guidelines presented
in~\cite{b1} and map them on our porting process. However, we do not plan to
restart the porting processs while mapping these guidelines, instead we want to
have a discussion and draw conclusions based on them.
