\section{Porting IxOS on ARM Boards} \label{sec:portingIxos}

To continue our study of evaluating the porting costs, we chose to port a large
scale system used for network testing. In this section we will present the
process of porting this system. 

\subsection{Porting Architecture}

The components of the porting architecture, presented in
Figure~\ref{fig:ixos_arch}, are:
\begin{itemize}
    \item A client for printing network testing data (\textbf{IxExplorer})
    \item A middleware for connecting the client to the machines that run the
    network testing suite (\textbf{Chassis})
    \item The actual machines that run the testing (\textbf{Cards})
    \item The device that benefits from network testing (\textbf{Device Under Test} - DUT)
\end{itemize}

IeExplorer is a Windows application that allows the user to connect to the
machines that run the testing, configure and run tests on them, and retrieve
information about the status of the tests.

To allow IxExplorer to easily connect to multiple machines that run the testing,
a Chassis is used. The most important application that runs on the Chassis is
IxServer. It creates a communication channel between IxServer and the testing
machines. 

Next, the cards are the most interesting part for our porting. They run on a
custom Ixia solution, IxVM to achieve the best performance for network testing. On top
of IxVM is placed IxOS Linux, a modified version of Linux with custom
device drivers, kernel parameters and userspace applications. The two most
interesting applications for our porting are InterfaceManager (IM) and IxStack.
Their role in the system is to manage the network interfaces, represented by
Port01 and Port02 in Figure~\ref{fig:ixos_arch}, and to load the network
protocols for the testing suites.

The device device under test is the consumer of the network traffic produced by
the cards. Usually it is connected between two \textbf{Ports} that will monitor the
behavior of the DUT while receiving traffic.

\begin{figure}
    \centering
    \def\svgscale{0.95}
    \input{fig/ixos_arch.pdf_tex}
    \caption{IxOS porting architecture}
    \label{fig:ixos_arch}
    \medskip
    \small
    The right part of the figure presents the modification of the card
    environment.
\end{figure}

\subsection{New IxOS Architecture}

In Figure~\ref{fig:ixos_arch} we highlighted the changes we made to the initial
architecture. First, we changed the hardware a card runs on. From IxVM we moved
to a Raspberry Pi 4. On top of the Raspberry Pi, we put a Raspberry Pi OS Linux
instead of IxOS Linux.

These changes required us to port InterfaceManager and IxStack to the new
Raspberry Pi environment so that we could benefit from the same functionality as in the
old Ixia custom hardware + IxOS Linux environment. The process of porting the
two applications is described in the next subsection.

To recreate the required environment for IM and IxServer, we had to install
custom tools, named pipes and shared libraries in the target environment.

\subsection{Previous work in this area}

Before we started the project, there existed an attempt to make InterfaceManager
and other applications independent of IxOS. The previous project focused on
compiling the whole IxOS infrastructure for x86 and then extract the binaries
for InterfaceManager and other relevant applications.

This helped us during our porting because the result of the previous work was a
portable application that could easily be moved in another Linux environment.
Making InterfaceManager independent of IxOS meant that all the assumptions made
by the application regarding the operating system interface were removed (e.g.,
custom device drivers and proc entries) and the work of porting it to another
environment was simply a task of finding the right tools for building the
binaries and solving unknown inconsistencies.

InterfaceManager had already a high degree of portability as it was written
in C++ using an object oriented paradigm. The architecture dependent code was
separated using constructs as ifdefs and the coding style was compiler agnostic,
meaning that we were able at any time to plug another compiler and generate the
correct binaries.

\subsection{The Porting Process}

The porting process consisted in moving the IxOS testing infrastructure from its
source environment to a new target environment consisting of an off-the-shelf
ARM board, Raspberry Pi, running a stock Linux operating system. When we started
the porting we set the following porting milestones:
\begin{itemize}
    \item Separate InterfaceManager from IxOS in a git repository
    \item Run InterfaceManager in QEMU
    \item Run InterfaceManager on Raspberry Pi hardware
\end{itemize}
We achieved all our proposed milestones during the twelve weeks of the project.
At the end of the project we divided the work into three logical stages
that cover the process of achieving the above milestones.

\subsubsection{Porting Stages}

The logical stages of our porting process are the following:
\begin{itemize}
    \item Build binaries for ARM64
    \begin{itemize}
        \item Separate IM + HostProxy from IxOS infrastructure
        \item Integrate IxStack with IM
        \item Build plugins for IM
        \item Test initial builds in QEMU
    \end{itemize}
    \item Create Card environment on RPi
    \begin{itemize}
        \item Run setup on RPi hardware
        \item Modify IM to run in the new environment
        \item Debug IM init time issues
    \end{itemize}
    \item Bring up ports on Chassis
    \begin{itemize}
        \item Find the cause of the IM init time issues
        \item Solve the problems regarding the link state of the card
    \end{itemize}
\end{itemize}
These stages with their respective substages were not completed in chronological
order. The process of completion was rather incremental, meaning that for
example we had to test the initial builds in QEMU each time we made a
modification in the \textit{Integrate IxStack in IM} stage, going back and forth
between the two substages.

In the first stage, we started the work of porting by trying to separate the
relevant components from IxOS and build them for ARM64. We started with
InterfaceManager and HostProxy, which were the building blocks of our porting,
and continued with IxStack and its components. The hardware was not available
at this time so we tested our builds using an emulator for the target
architecture (i.e., QEMU).

Next, when the hardware arrived, we used the Raspberry Pi to test our builds.
This proved to be a great advantage for us because the testing environment
worked better on hardware than in the emulator. After we built all our
components, we started to focus on the inconsistencies between the source
environment and the target environment. For that we had to modify some parts of
the program that were source environment dependent. However the porting did not
run as smoothly as we expected. We had to investigate multiples paths to
understand the problems we faced, namely the problems regarding the
initialization of our Card.

In stage three we investigated the initialization problems from another
perspective. This time we investigated what was the initialization sequence on
IxServer. Here we discovered that IxServer did not set the link state correctly
when connecting to our Card. To solve that we tracked the variables and code
zones that modify the link state in order to find out which code lines cause
trouble. In the end we discovered that a problem regarding the initialization
of the protocol loader was causing our start-up issues. After we solved
these issues we were able to fully port InterfaceManager on Raspberry Pi and
integrate it with the bigger system consisting of IxServer and IxExplorer.

\subsubsection{Porting Difficulties}

Each stage came with its own difficulties that we needed to solve in order to
move to the next stages. In this section we will present the most relevant
impediments we faced in each stage.

\paragraph{Build binaries for ARM64}

The first and most difficult impediment we faced when trying to move the build
system from the source environment to ARM64 was to understand the architecture
and design of the code and the build system. The documentation was lacking so
we had to rely on our experience with the system in order to understand the
peculiarities of each component. We did not spend much time on understanding the
code architecture, rather we spent time on understanding the build system
because we wanted to deliver working binaries for ARM64 as soon as possible.

The build system used SCons, rather than classic Makefile, which required us to
use a system that ran python2 out of the box or install it using unconventional
methods because python2 is no longer supported on most systems. After we
modified the build system and generated binaries for ARM64 we had to test them
in an emulated environment because hardware was not available at that time. Our
solution was to use QEMU which caused us problems while testing the builds.
The speed of the operating system running inside QEMU was slowing us down. We
had to wait a considerable time for installing our builds on the
machine, running the ported application and analyzing the output so that we
could solve the eventual inconsistencies.

After the hardware arrived, we were able to test our build in a faster
environment.

\paragraph{Create Card environment on RPi}

At this time we had to prepare the target environment for a more advanced
type of testing. Until this point we were solving start-up issues with regards
to InterfaceManager. Now we had to prepare the communication environment with
other components as IxServer and IxExplorer. Before placing our Raspberry Pi
environment in the same network as IxServer we had to work with a clumsy
configuration: IxServer ran in a Virtual Box machine and the Raspberry Pi was
connected to the computer that ran VirtualBox. To connect the two we had to make
a bridge between the Virtual Box network and the host network. This did not work
as expected because this configuration broke the IxServer machine.

When we succeeded in putting the Raspberry Pi in the same network as IxServer,
we faced problems with the Linux operating system running on Card (i.e.,
Raspberry Pi OS). dmidecode(8) did not work on this system because the operating
system did not allow us to access /dev/kmem. Another problem we faced featured
the assumptions InterfaceManager made about the environment it ran in. In its
source environment it relied on many Ixia custom device drivers and ioctl's.
In the target environment, these interfaces did not exist so we had to
skip their usage.

After we solved all the problems regarding the interaction of InterfaceManager
with the new environment, we stepped in stage three where we solved the
communication problems of InterfaceManager with the other components (i.e.,
IxServer and IxExplorer).

\paragraph{Bring up ports on Chassis}

In this stage the most difficult impediment we faced was figuring out what went
wrong when IxServer was communicating with Card. We had no reference
communication so we had to rely on trial and error methods to find out what went
wrong.

Firstly, we tried to investigate if the problem was with Card stats. We assumed
that the Card did not initialize correctly when communicating with IxServer
because it did not send periodical statistics. However this proved to be false,
but it came with a rather huge cost. We spent approximately three days trying
to see if the problem was the statistics.

Secondly, we focused on the code in IxServer, leaving behind the Card. The code
is written in C++, using an object oriented paradigm. The coding style helped us
to surf faster through the code and filter unimportant code zones. However the
difficulty here was the code size. We had to read multiple thousand-lines files
in order to find the communication problem. Because of time reasons we only
patched the code in IxServer side, without modifying the Card.
