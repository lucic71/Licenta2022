\section{Background} \label{sec:background}

The process of reusing code in a new environment has clear advantages over
rewriting. However this process does not come cheap, as the task of reusing code
through porting is time consuming.

\subsection{Porting and porting tasks}

Porting is the act of producing an executable version of a software unit or
system in a new environment based on an existing version~\cite{b7}. This is
seldom an easy task because it involves a good amount of code refactoring and
rewriting. It can be avoided, however, if the original design has portability
incorporated by using constructs as multi-platform libraries, modular code or
standard compiler behaviors.

The environment is defined as the set of software and hardware elements that
interact with the system. This includes, but it is not restricted to: operating
system, communication methods, configuration files and system variables,
hardware architecture or human interaction. 

Mooney~\cite{b7} presents two components of the porting process: transportation
and adaptation. The first is described as the act of moving the system (code or
binary executable) to a new environment and the latter is described as the act
of modifying the system in order to be compatible with the new environment.
Transportation is facilitated by communication channels to the target
environment, both online and physical. Adaptation consumes more development
resources than transportation because it implies translating the source code to
the new environment, solving possible inconsistencies and making sure that the
software system behaves in a well defined manner when ran in the new
environment.

Mooney's model is very simplistic regarding the tasks that can occur during
a porting process. From his model, and from the work of Hakuta and Ohminami~
\cite{b2}, and Tanaka et al.~\cite{b1} a more accurate model can be created that
reflects better the components involved in porting an application. We also added
our input in this model based on the needs of our project. Finally, the
improvised model of porting has the following structure:
\begin{itemize}
    \item Advance preparations
        \begin{itemize}
            \item Surveying development environment
            \item Surveying target OS
            \item Surveying program for porting
            \item Surveying documentation
            \item Adjusting development environment
            \item Adjusting target environment
            \item Initial source code modifications
        \end{itemize}
    \item Building for target environment
        \begin{itemize}
            \item Build system triggering and modification
            \item Installation on remote environment
            \item Reviewing inconsistencies between source and remote
            environments
            \item Solving problems with external dependencies
        \end{itemize}
    \item Testing
        \begin{itemize}
            \item Testing in simulated environment
            \item Testing in target environment
        \end{itemize}
    \item General duties
        \begin{itemize}
            \item Documentation
            \item Progress tracking
            \item Discussions
        \end{itemize}
\end{itemize}

In \textit{Advance preparations} the developer familiarizes with the tools,
environments and the program to be ported, and also adjusts the development and
target environments for creating and testing the program to be ported. Finally,
if needed, the developer also makes \textit{Initial source code modifications}
that reflect the modification of the source environment to the new target
environment (e.g., modification in system call numbers and error
numbers~\cite{b22}).

In the second task of \textit{Building for target environment} the developer
focuses on three issues: compiling the code to generate binaries for the
target environment, installing the code in the target environment and solving
the inconsistencies between the source and target environment.

The previous task and \textit{Testing} are the core of the porting process, they
deliver the ported application that operates in the target environment.
Testing is of two types in this model. The application can either be tested in a
simulated environment for convenience (e.g., hardware is not available at the
moment of testing) or can be tested directly in the target environment.
Furthermore this tasks includes implicitly the time allocated for setting up the
testing environment.

The last task, that is \textit{General duties}, encompasses subtask related to
human interaction activities. In this part of the project the developer focuses
on delivering documents that describe the process of porting or other
information relevant to the project and focuses on planning and discussing
aspects with regards to difficulties encountered during the process.

\subsection{Porting costs}

Now that we have defined what the porting process is and what are the relevant
porting tasks, let us describe the porting costs associated with the process.

Porting costs are measured in man-hours~\cite{b1, b2}. While the costs are
determined by program size and contents~\cite{b2}, other factors as portability
impediments, human factors or environmental factors~\cite{b2} play a
considerable role. To understand the factors that influence the porting costs,
these factors are quantified in indices that describe how much of an influence
they have.

In our work we will use three indices of this type as follows: portability
impediments index, human factors index and environmental factors index.

The first index, portability impediments, answers the following question: how
portable was the program to be ported and how many difficulties did the
developer meet in the porting process? The factors that influence this index are
described in (S1$\sim$S11~\cite{b2}) and the index is computed using the
following formula: \[ \alpha_p = \eta * \sum_{n=1}^{12} \omega_i S_i \]

Here $\eta$ is a portability design index, $\omega_i$ is the weight assigned to
each factor and $S_i$ is 1 when the impediment factor $i$ exists, otherwise is
0.

The second index, human factors, answers the following question: what role did
the experience and knowledge of the developer play in the porting process? The
factors that influence this index are described in (H1$\sim$H5~\cite{b2}) and
the index is computed using the following formula: \[ \sum_{n=1}^{5} H_i \]

Here $H_i$ are the human factors presented in~\cite{b2}. Their values range
from -2 which reflects the maximum productivity while 2 reflects the minimum
productivity.

The third index, environmental factors, answers the following question: how did
the development and testing environments, and the tools used during the porting
process affect the porting costs? The factors that influence this index are
described in (E1$\sim$E3~\cite{b2}) and the index is computed using the
following formula: \[ \sum_{n=1}^{3} E_i \]

Here $E_i$ are the following environmental factors as presented in~\cite{b2}:
\begin{itemize}
    \item Development environment (E1)
    \item Unit test environment (E2)
    \item System test environment (E3)
\end{itemize}

As for H1$\sim$H5, the values for E1$\sim$E3 range between -2 and 2, -2 being the best
score for $E_i$, while 2 being the worst.
