\section{Discussions on Porting Costs}

% What I want to say in this section:
%   - I want to reflect on imporving the porting process we did. Tanaka proposes
%   the following seven ways of raising porting efficiency:
%     - porting guidlines
%     - porting compatibility checking tool
%     - portability evaluation tool
%     - tool for generating system calling routines
%     - program structure viewing tool
%     - os emulator
%     - test support tools
%   - I would like to see how relevant are they for our project. there is no way
%   at the moment to actually test the ways of improving porting efficiency so
%   I will keep the statements at the level of discussion, thinking about how
%   would have the process been different if we used one of the above ways (this
%   is chapter 3.2 from Tanaka)

% Subjects for discussion:
%  - the porting tasks affect each other in a non linear fashion
%  - we had errors in our extraction of costs from tracking
%  - it would be nice to have a dependency graph for the porting tasks. however
%  we could try to sketch some dependencies and see if an imporvement on one
%  side affects the other side
%  - what where the limitations in implementations and how do the improvements
%  from Tanaka apply to them
%  - compare our porting costs with the Table from Tanaka

In this section we present observations about the porting process and the
porting costs, focusing on the limitations and open discussions in the area of
porting based on our experience of porting IxOS infrastructure.

Let us start by discussing the limitations of the porting model we proposed in
section~\ref{sec:background}. The model assumes that the tasks are executed in
sequential order, which is not true. The tasks are rather executed in a
non-linear fashion. For example while building for the target environment,
installing the binaries in this environment and testing the ported application,
we also had discussions about the difficulties and errors we encountered so that
we could later come back and solve the inconsistencies. However there was no
easy way to describe this dynamic so we chose to represent the task as they
would come one after another. The non-linearity of this porting model causes
problems while computing the porting costs. If one wants to compute the costs
with minimum error it means that more time must be allocated to the
\textit{Progress tracking} subtask. This strategy might not bring the best
results if the fixed total time of porting is computed in advance because
allocating more time for progress tracking means that other porting tasks must
have lesser time allocated. A good strategy here would compute first the
accepted error of man-hours present in the final porting costs so that an
acceptable ammount of time is allocated for tracking the progress of the
project. For our porting process it was not vital to have a small error in the
porting costs, we were interested to see an approximative distribution of time
per tasks so that we could answer questions as: where did we spend most time and
why? or what was the relation between development and testing? For this we
allocated an hour each week for tracking the status of the project, the planned
objectives for the future and the major difficulties we met in the respective
week.
