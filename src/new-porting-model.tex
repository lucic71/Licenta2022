\chapter{A revised porting model} \label{sec:revised-porting-model}

Given that the model of porting presented in Chapter~\ref{sec:background} was
crafted for the particular use case of one project~\cite{tanaka} and we wanted
to use a more general porting model, we revised the old porting model and
modified it so that it could match more porting projects.

In the revised model, we keep the \textit{General duties} and \textit{Advance
preparations} tasks and modify the \textit{Workstation testing} and
\textit{Target testing}. We do these changes because we want to emhpasize the
allocated time between testing and development with the \textit{Building for
target environment} and \textit{Testing} tasks.

Following is the revised model:
\begin{itemize}
    \item Advance preparations
        \begin{itemize}
            \item Surveying development environment
            \item Surveying target OS
            \item Surveying program for porting
            \item Surveying documentation
            \item Adjusting development environment
            \item Adjusting target environment
            \item Initial source code modifications
        \end{itemize}
    \item Building for target environment
        \begin{itemize}
            \item Build system triggering and modification
            \item Installation on remote environment
            \item Reviewing inconsistencies between source and remote
            environments
            \item Solving problems with external dependencies
        \end{itemize}
    \item Testing
        \begin{itemize}
            \item Testing in simulated environment
            \item Testing in target environment
        \end{itemize}
    \item General duties
        \begin{itemize}
            \item Documentation
            \item Progress tracking
            \item Discussions
        \end{itemize}
\end{itemize}

In \textit{Advance preparations} the developer familiarizes with the tools,
environments and the program to be ported, and also adjusts the development and
target environments for creating and testing the program to be ported. Finally,
if needed, the developer also makes \textit{Initial source code modifications}
that reflect the modification of the source environment to the new target
environment (e.g., modification in system call numbers and error
numbers~\cite{callahanopenbsd}).

In the second task of \textit{Building for target environment} the developer
focuses on three issues: compiling the code to generate binaries for the
target environment, installing the code in the target environment and solving
the inconsistencies between the source and target environment.

The previous task and \textit{Testing} are the core of the porting process, they
deliver the ported application that operates in the target environment.
Testing is of two types in this model. The application can either be tested in a
simulated environment for convenience (e.g., hardware is not available at the
moment of testing) or it can be tested directly in the target environment.
Furthermore this tasks includes implicitly the time allocated for setting up the
testing environment.

The last task, that is \textit{General duties}, encompasses subtasks related to
human interaction activities. In this part of the project the developer focuses
on delivering documents that describe the process of porting or other
information relevant to the project and focuses on planning and discussing
aspects with regards to difficulties encountered during the process.

