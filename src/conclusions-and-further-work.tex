\chapter{Conclusions and Further Work} \label{sec:conclusions}

\section{Conclusions}

\lm{Comment on structure: here, you can eliminate the {\textbackslash}section tag and begin with something along the lines of: ``This thesis presents the following work:'' [...]. You can still leave ``Further work'' in a separate section though.}

We succeeded in porting the IxOS infrastructure on ARM boards. We separated the
relevant components for our porting (i.e., InterfaceManager and IxStack) from
IxOS so that they can be run on every~\lm{on any?} ARM-based Linux distribution. At the
moment of writing the project is in the proof-of-concept stage. If there is
interest for continuing the project or integrating it in other projects inside
the company, we provided the necessary environment for deploying it.~\lm{Not sure that the last two sentences are needed at all.}

We have extracted the porting costs for porting IxOS infrastructure on ARM
boards. Thus we understood what the weaknesses and the strengths of our project
were. The lack of understanding of the project structure and project use cases
proved itself to be an important factor during the process of porting. Because
of this reason we had to spend additional time on testing the system and
understanding its components. This time might have been better allocated on
solving problems and inconsistencies. A strength of our project was the fact
that the program for porting~\lm{the ported program?} had a high degree of portability. This helped us to
shrink the volume of inconsistencies between the source environment and the
target environment.

We succeeded in creating a better~\lm{In what sense is it ``better''? is it more accurate? is it more comprehensive? try to be more specific.} model for software porting starting from the
model presented in~\cite{hakuta,tanaka} and from our project specific needs. We
contributed to this model by making it more generic and allowing other software
porting projects to easily map their needs on this model.~\lm{Ah, I see. I think you should just say this, and also briefly say why we needed to extend their model.}

Finally, we have provided a discussion on the limitations of the model we
created and provided a comparison between our porting results and the results
presented in the first work~\cite{hakuta} that came up with the model we used as a
basis for our generic porting model.

\section{Further Work}

We plan to test the performance of the system so that we can achieve the other
goal we proposed in the beginning of this work~\lm{Rather, I would formulate this as such: we want to explore other aspects of our port to ARM.} (i.e., explore Ixia testing
infrastructure on other architectures so that we can achieve a better
performance of the system). To achieve this goal we will compare our solution in
the target environment with the same solution in the source environment. We
expect to see better results in the new environment for some network testing
suites than in the old environment. Furthermore, we plan to compare our solution
with other open-source network testing tools.~\lm{Ixia is already better than open-source network testing tools, so... I don't know.}

To complete the analysis of the factors that affected the porting costs we plan
to analyze the characteristics of the program to be ported. We want to analyze
the program size and contents and the content of the changes needed for porting.
By doing this we hope~\lm{rather, we aim?} to find a direct correspondence between the program to be
ported and specific porting subtasks (e.g., \textit{Solving inconsistencies
between source and target environment}).

Finally, we want to analyze the porting improvements guidelines presented
in~\cite{hakuta} and map them on our porting process. However, we do not plan to
restart the porting process while mapping these guidelines, instead we want to
have a discussion and draw conclusions based on them.

\lm{I think there is a more general question to be answered here for future work (and it might have been largely solved by the IxOS team, I don't know): can this protocol/control plane infrastructure be provided as a drop-in component for any Linux-based system? regardless of the target architecture, supposing a C++ compiler exists for it; and regardless of the specifics of the target Linux environment, supposing it's standard Linux. If this is possible, then it's a big win for Ixia, since they can deploy the control plane tester anywhere (cloud, commodity boards, other systems that we haven't thought of yet). Not sure whether to mention this explicitly here, but it is certainly worth mentioning that we want the testing infrastructure to be portable to as many environments (OS, compiler, architecture) as possible.}
