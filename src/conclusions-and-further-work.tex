\chapter{Conclusions} \label{sec:conclusions}

We succeeded in porting the IxOS infrastructure on ARM boards. We separated the
relevant components for our porting (i.e., InterfaceManager and IxStack) from
IxOS so that they can be run on any ARM-based Linux distribution. At the
moment of writing the project is in the proof-of-concept stage. If there is
interest for continuing the project or integrating it in other projects inside
the company, we provided the necessary environment for deploying it.

We have extracted the porting costs for porting IxOS infrastructure on ARM
boards. Thus we understood what the weaknesses and the strengths of our project
were. The lack of understanding of the project structure and project use cases
proved itself to be an important factor during the process of porting. Because
of this reason we had to spend additional time on testing the system and
understanding its components. This time might have been better allocated on
solving problems and inconsistencies. A strength of our project was the fact
that the the ported program had a high degree of portability. This helped us to
shrink the volume of inconsistencies between the source environment and the
target environment.

We succeeded in creating a more accurate and comprehensive software porting
model with regards to today's standards of software engineering starting from
the model presented in~\cite{hakuta,tanaka} and from our project specific needs.
We contributed to this model by making it more generic and allowing other
software porting projects to easily map their needs on this model.

Finally, we have provided a discussion on the limitations of the model we
created and provided a comparison between our porting results and the results
presented in the first work~\cite{hakuta} that came up with the model we used as a
basis for our generic porting model.

\chapter{Further Work}

We aim to make the testing infrastructure portable to as many environments (OS,
compiler, architecture) as possible. This would be a big win for the software
project because people interested in using the portable software parts as
drop-in components on any Linux-based system could do it with little effort,
supposing that a C++ compiler would exists for the specific architecturure the
software will be installed on.

We plan to explore other aspects of our port to ARM, as system performance of
while running network testing suites.  To achieve this goal we should compare
our solution in the target environment with the same solution in the source
environment. We expect to see better results in the new environment for some
network testing suites than in the old environment. Furthermore, we plan to
compare our solution with other open-source network testing tools.

To complete the analysis of the factors that affected the porting costs we plan
to analyze the characteristics of the program to be ported. We want to analyze
the program size and contents and the content of the changes needed for porting.
By doing this we aim to find a direct correspondence between the program to be
ported and specific porting subtasks (e.g., \textit{Solving inconsistencies
between source and target environment}).

Finally, we want to analyze the porting improvements guidelines presented
in~\cite{hakuta} and map them on our porting process. However, we do not plan to
restart the porting process while mapping these guidelines, instead we want to
have a discussion and draw conclusions based on them.
