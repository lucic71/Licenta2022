\chapter{Introduction}\pagestyle{fancy}

Porting a software system and evaluating the porting costs is a hard problem in
systems programming.~\cite{b1,b2,b3,b4,b9,b10,b11,b12,b13,b14,b15,b16} The
reasons for porting software systems are various: the developers want to enhance
the performance, the hardware environment starts to get deprecated or the system
wants to take advantage of features unavailable in the current environment.

Software development costs and invested resources for understanding, maintaining
and developing new features for a system are big~\cite{b17,b18,b19}. We would
like to preserve these investments when the need for a new environment arises.
To achieve this goal people designed programming languages and compilers that
increased portability, operating systems that could run on multiple hardware
platforms~\cite{b16} and various standards that allowed developers to talk to
the computer using well defined interfaces~\cite{b20}. However the porting
process is a non-trivial endeavor up to this day, it is error prone and time
consuming.

As with other software development processes, porting has its own costs
associated. It is important to evaluate them and understand their implications
in the project so that the developer can optimize the process of porting in the
future and call attention to the weaknesses and strengths of the process. In
this work we describe the experience of porting the IxOS testing infrastructure,
used by Ixia for high performance network testing, on ARM off-the-shelf boards.
We do this for two reasons. Firstly, we are interested in exploring Ixia testing
infrastructure on newer environments with the hope that we will improve the
performance of the system. Secondly, we are interested in evaluating porting
costs related to the project for the reasons listed in the beginning of this
paragraph. 

We make the following contributions in this work: we review the porting models
described in~\cite{b1,b2,b9} and propose a more general model that can be
applied to modern software porting, we review the porting costs factors
described in~\cite{b2} and analyze their relevance in our porting work, and we
provide guidelines and discussions on the topic of improving software porting
based on the experience of porting IxOS infrastructure. Furthermore, we
separated the relevant parts for our porting from the infrastructure and ported
them on ARM architecture. By doing this we allow the the developing of further
testing tools and systems on ARM boards by Ixia.

This paper highlights the porting issues associated with porting the IxOS
network testing infrastructure, both the hardware and software components, to
ARM off-the-shelf boards, Raspberry Pi 4 in our case. In
Section~\ref{sec:background} we present different terms associated with porting:
what porting is, a porting model, porting costs and factors. In
Section~\ref{sec:portingIxos} we divide our work in multiple steps. For each
step we present a description, the targets and milestones, and the final
results. After the porting process is discussed, we highlight in
Section~\ref{sec:eval} the porting costs associated with our work, including an
analysis of the factors that influenced the presented costs. Finally, in
Section~\ref{sec:discussion} we conduct a discussion about the porting process
and the porting costs where we investigate porting difficulties, observations
about the porting tasks and ways to improve the porting process. 
