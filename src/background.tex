\chapter{The anatomy of the porting process} \label{sec:background}

The process of reusing code in a new environment has clear advantages over
rewriting~\cite{frakes1995sixteen}. However this process does not come cheap, as
the task of reusing code through porting is time consuming. In this chapter we
present the theoretical background and the related work in the field of porting
including the porting process with its tasks, porting costs and porting factors.

\section{Porting and porting tasks}

Porting is the act of producing an executable version of a software unit or
system in a new environment based on an existing
version~\cite{mooney1990strategies}. This is seldom an easy task because, in
general, it involves a good amount of code refactoring and rewriting. It can be
avoided, however, if, in particular, the original design has portability
incorporated by using, for example, constructs as multi-platform libraries,
modular code or standard compiler behaviors~\cite{tanenbaum1978guidelines}.

The environment is defined as the set of software and hardware elements that
interact with the system. This includes, but it is not restricted to: operating
system, communication methods, configuration files and system variables,
hardware architecture or human interaction. Two software environments involved
in the porting process will never be the same because of the software and
hardware inconsistencies. On one hand, programs between operating systems will
not work, even if the hardware is the same. For example MacOS uses MACH for
executable files while Linux uses ELF, moreover even if they follow the POSIX
standard, they may have implementation details that do not align with each
other. On the other hand, processor architectures vary from one another in the
way they understand machine language and even if they do not vary, custom
hardware attached to these processors may make porting difficult as the software
must be rewritten in order to accomodate the new peripherals.

Mooney~\cite{mooney1990strategies} presents two components of the porting process:
transporation and adaptation. The first is described as the act of moving the
system (code or binary executable) to a new environment and the latter is
described as the act of modifying the system in order to be compatible with the
new environment. Transportation is facilitated by communication channels to the
target environment, either online (file sharing systems, remote connections) or
physical (using data storage devices). Adaptation consumes more development
resources than transportation because it implies translating the source code to
the new environment, solving possible inconsistencies and making sure that the
software system behaves in a well defined manner when ran in the new
environment.

Mooney's model is very simplistic regarding the tasks that can occur during a
porting process. Hakuta and Ohminami~ \cite{hakuta}, and Tanaka et
al.~\cite{tanaka} created a more accurate model that reflects better components
involved in porting an application. The tasks involved in their model are the
following:
\begin{itemize}
    \item Advance preparations
        \begin{itemize}
            \item Surveying development environment
            \item Surveying OS
            \item Surveying program for porting
            \item Surveying workstation development environment
            \item Adjusting target environment
            \item Initial source code modifications
        \end{itemize}
    \item Workstation testing
        \begin{itemize}
            \item Standalone testing on workstation
            \item Linked testing on workstation
        \end{itemize}
    \item Target testing
        \begin{itemize}
            \item File-making
            \item File system creation
            \item Installation on target
            \item Test program creation
            \item Linked test on target
        \end{itemize}
    \item General duties
        \begin{itemize}
            \item Documentation
            \item Progress tracking
            \item Discussions
        \end{itemize}
\end{itemize}

They emphasize on spending additional time on getting familiar with the system
and only then starting to port the application per se. Testing is conducted in a
workstation environment (that is, testing the builds on the local machine or in
a local simulator/emulator) and in the target environment. Finally they also
include non-technical duties as documentation, tracking and discussions.

\section{Porting costs and factors}

Porting costs, and more generally, software development costs, are measured in
man-hours~\cite{tanaka, hakuta}. While the costs are determined by program size
and contents~\cite{hakuta}, other factors as portability impediments, human
factors or environmental factors~\cite{hakuta} play a considerable role. To
understand the factors that influence the porting costs, these factors are
quantified in indices that describe how much of an influence they have.

In our work we will use three indices of this type as follows: portability
impediments index, human factors index and environmental factors index.

The first index, portability impediments, answers the following question: how
portable was the program to be ported and how many difficulties did the
developer meet in the porting process? The factors that influence this index are
described in (S1$\sim$S11~\cite{hakuta}) and the index is computed using
Equation~\ref{eq:pii}.


\begin{equation} \label{eq:pii}
\alpha_p = \eta * \sum_{n=1}^{12} \omega_i S_i
\end{equation}

Here $\eta$ is a portability design index, $\omega_i$ is the weight assigned to
each factor and $S_i$ is 1 when the impediment factor $i$ exists, otherwise is
0. The factors are placed in three categories: differences in processor
architecture, OS disparity and differences in language processor.

The second index, human factors, answers the following question: what role did
the experience and knowledge of the developer play in the porting process? The
factors that influence this index are described in (H1$\sim$H5~\cite{hakuta}) and
the index is computed using Equation~\ref{eq:hfi}.
\begin{equation} \label{eq:hfi}
\sum_{n=1}^{5} H_i
\end{equation}

Here $H_i$ are the human factors presented in~\cite{hakuta}. Their values range
from -2 which reflects the maximum productivity while 2 reflects the minimum
productivity.

The third index, environmental factors, answers the following question: how did
the development and testing environments, and the tools used during the porting
process affect the porting costs? The factors that influence this index are
described in (E1$\sim$E3~\cite{hakuta}) and the index is computed using
Equation~\ref{eq:hfi}.

\begin{equation} \label{eq:hfi}
\sum_{n=1}^{3} E_i
\end{equation}

Here $E_i$ are the following environmental factors as presented in~\cite{hakuta}:
\begin{itemize}
    \item Development environment (E1)
    \item Unit test environment (E2)
    \item System test environment (E3)
\end{itemize}

As for H1$\sim$H5, the values for E1$\sim$E3 range between -2 and 2, -2 being the best
score for $E_i$, while 2 being the worst.

After we described the anatomy of the porting process and presented the porting
costs associated with this process, we present a practical example of porting a
software system and evaluating the costs.
