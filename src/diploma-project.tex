\documentclass[12pt,a4paper]{report}

\usepackage[utf8]{inputenc} % pentru suport diacritice
\usepackage[romanian]{babel} % setări pentru limba română
\renewcommand\familydefault{\sfdefault} % sans serif

\usepackage[margin=2.54cm]{geometry}	% dimensiuni pagină și margini
\usepackage{graphicx} % support the \includegraphics command and options

% formatting sections and subsections
\usepackage{textcase}
\usepackage[titletoc, title]{appendix}
\usepackage{titlesec}
\titleformat{\chapter}{\large\bfseries\MakeUppercase}{\thechapter}{2ex}{}[\vspace*{-1.5cm}]
\titleformat*{\section}{\large\bfseries}
\titleformat*{\subsection}{\large\bfseries}
\titleformat*{\subsubsection}{\large\bfseries}

\usepackage{chngcntr}
\counterwithout{figure}{chapter} % no chapter number in figure labels
\counterwithout{table}{chapter} % no chapter number in table labels
\counterwithout{equation}{chapter} % no chapter number in equation labels

\usepackage{booktabs} % for much better looking tables
\usepackage{url} % Useful for inserting web links nicely
\usepackage[bookmarks,unicode,hidelinks]{hyperref}

\usepackage{array} % for better arrays (eg matrices) in maths
\usepackage{paralist} % very flexible & customisable lists (eg. enumerate/itemize, etc.)
\usepackage{verbatim} % adds environment for commenting out blocks of text & for better verbatim
\usepackage{subfig} % make it possible to include more than one captioned figure/table in a single float
\usepackage{enumitem}
\setlist{noitemsep}

%%% HEADERS & FOOTERS
\usepackage{fancyhdr}
\pagestyle{empty}
\renewcommand{\headrulewidth}{0pt}
\renewcommand{\footrulewidth}{0pt}
\lhead{}\chead{}\rhead{}
\lfoot{}\cfoot{\thepage}\rfoot{}

\newcommand{\HeaderLineSpace}{-0.25cm}
\newcommand{\UniTextRO}{UNIVERSITATEA POLITEHNICA DIN BUCUREȘTI \\[\HeaderLineSpace]
FACULTATEA DE AUTOMATICĂ ȘI CALCULATOARE \\[\HeaderLineSpace]
DEPARTAMENTUL DE CALCULATOARE\\}
\newcommand{\DiplomaRO}{PROIECT DE DIPLOMĂ}
\newcommand{\AdvisorRO}{Coordonator științific:}
\newcommand{\BucRO}{BUCUREȘTI}

\newcommand{\UniTextEN}{UNIVERSITY POLITEHNICA OF BUCHAREST \\[\HeaderLineSpace]
FACULTY OF AUTOMATIC CONTROL AND COMPUTERS \\[\HeaderLineSpace]
COMPUTER SCIENCE AND ENGINEERING DEPARTMENT\\}
\newcommand{\DiplomaEN}{DIPLOMA PROJECT}
\newcommand{\AdvisorEN}{Thesis advisor:}
\newcommand{\BucEN}{BUCHAREST}

\newcommand{\frontPage}[6]{
\begin{titlepage}
\begin{center}
{\Large #1}  % header (university, faculty, department)
\vspace{50pt}
\begin{tabular}{p{6cm}p{4cm}}
\includegraphics[scale=0.8]{pics/upb-logo.jpg} &
	\includegraphics[scale=0.5,trim={14cm 11cm 2cm 5cm},clip=true]{pics/cs-logo.pdf}
\end{tabular}

\vspace{105pt}
{\Huge #2}\\                           % diploma project text
\vspace{40pt}
{\Large #3}\\ \vspace{0pt}  % project title
{\Large #4}\\                          % project subtitle
\vspace{40pt}
{\LARGE \Name}\\                   % student name
\end{center}
\vspace{60pt}
\begin{tabular*}{\textwidth}{@{\extracolsep{\fill}}p{6cm}r}
&{\large\textbf{#5}}\vspace{10pt}\\      % scientific advisor
&{\large \Advisor}                                    % advisor name
\end{tabular*}
\vspace{20pt}
\begin{center}
{\large\textbf{#6}}\\                                % bucharest
\vspace{0pt}
{\normalsize \Year}
\end{center}
\end{titlepage}
}

\newcommand{\frontPageRO}{\frontPage{\UniTextRO}{\DiplomaRO}{\ProjectTitleRO}{\ProjectSubtitleRO}{\AdvisorRO}{\BucRO}}
\newcommand{\frontPageEN}{\frontPage{\UniTextEN}{\DiplomaEN}{\ProjectTitleEN}{\ProjectSubtitleEN}{\AdvisorEN}{\BucEN}}

\linespread{1.15}
\setlength\parindent{0pt}
\setlength\parskip{.28cm}

%% Abstract macro
\newcommand{\AbstractPage}{
\begin{titlepage}
\textbf{\large SINOPSIS}\par
\AbstractRO\par\vfill
\textbf{\large ABSTRACT}\par
\AbstractEN \vfill
\end{titlepage}
}

%% Thank you macro
\newcommand{\ThanksPage}{
\begin{titlepage}
{\noindent \large\textbf{MULȚUMIRI}}\\
\Thanks
\end{titlepage}
}



%%%%%%%%%%%%%%%%%%%%%%%%%%%%%%%%%%%%%%%%%%%%%%%%%%
%%
%%          End of template definitions
%%
%%%%%%%%%%%%%%%%%%%%%%%%%%%%%%%%%%%%%%%%%%%%%%%%%%


%%% Puteți elimina aceste linii din lucrare, servesc numai pentru template.
\newcommand{\worktype}[1]{[\textit{#1}] }
\newcommand{\dezvoltare}{\worktype{Dezvoltare de produs}}
\newcommand{\cercetare}{\worktype{Cercetare}}
\newcommand{\ambele}{\worktype{Ambele}}
%%%


%%
%%   Campurile de mai jos trebuie modificate de autor. Modificati doar continutul, nu si numele fiecarei definitii
%%
\newcommand{\ProjectTitleRO}{Titlul proiectului de diplomă (ex: Șablon proiect de diplomă)}
\newcommand{\ProjectSubtitleRO}{Subtitlu (ex: versiunea 2018)}
\newcommand{\ProjectTitleEN}{Diploma Project Title  (eg: Diploma project template)}
\newcommand{\ProjectSubtitleEN}{Subtitle (eg: 2018 version)}
\newcommand{\Name}{Ioana Popescu}
\newcommand{\Advisor}{Prof. dr. ing. Andrei Ionescu}
\newcommand{\Year}{2018}

% Setări document
\title{Proiect de diplomă}
\author{\Name}
\date{\Year}

%%
%%   Campurile aferente rezumatului
%%
\newcommand{\AbstractRO}{

Pana in prezent portarea se bazeaza pe metode ad-hoc care au la baza factori ca experienta
dezvoltatorului sau claritatea sistemului ce se vrea portat. Acest studiu ofera o prezentare a
problemei portarii software bazandu-ne pe experienta practica de a porta infrastructura de testare Ixia
pe arhitectura ARM. Pentru acomodarea pe noua platforma am facut modificari ce tin de interfatarea
cu sistemul de operare sau de pastrarea compatibilitatii cu celelalte componente ale ecosistemului.
Testarea s-a desfasurat in mai trei etape: etapa functionala, etapa de performanta si
evaluarea procesului de portare. Rezultatele arata ca TBD. Aceste rezutlate faciliteaza formalizarea
unor tehnici de portare explicate pe larg de-a lungul lucrarii, cum ar fi impartirea si decuplarea
unitatilor sistemului sau izolarea bucatilor dependente de arhitectura.

Ce ar fi sa adaugam pe langa studiul nostru de IxOS si un studiu al portabilitatii Linux?
}

\newcommand{\AbstractEN}{
	bla bla
}


%%
%%   Campurile aferente paginii de multumiri
%%
\newcommand{\Thanks}{(opțional) Aici puteți introduce o secțiunea specială de mulțumiri / acknowledgments. }

\begin{document}

\frontPageRO
\frontPageEN

\begingroup
\linespread{1}
\tableofcontents
\endgroup

\AbstractPage

% poate fi comentata sau stearsa
\ThanksPage


% Textul licentei incepe de aici



\chapter{Introducere}\pagestyle{fancy}
% * <marios.choudary@gmail.com> 2018-02-28T11:38:18.106Z:
%
% > INTRODUCERE
% Am scos de aici referintele la font pentru a nu mai fi dependenti de Calibri. Personal, nici nu sunt sigur ca ajuta prea mult aceasta recomandare si mi se pare bun font-ul default din Latex (Computer Modern). Daca sunteti de-acord, va rog sa stergeti liniile comentate de mai jos, precum si cele referitoare la fontul Calibri din restul documentului.
%
% ^.
Parametrii de formatare recomandați pentru lucrare:
\begin{itemize}
 %\item Font recomandat: Calibri; Dimensiune font: 12;
 \item Dimensiune font: 12;
 \item Spațiere între linii: 1,5; Spațiere după paragraf: 8pt;
 \item Stil: Justified;
 \item Dimensiune pagină: A4; Margini: 2,54cm/ 2,54cm/ 2,54cm/ 2,54cm;
 %\item Heading1: Calibri, 14, bold, all caps;
 %\item Heading2: Calibri, 14, bold;
 %\item Heading3: Calibri, 12.
 %\item Font pentru formule: Cambria Math, 12.
 \item Heading1: 14, bold, all caps;
 \item Heading2: 14, bold;
 \item Heading3: 12.
 \item Font size pentru formule: 12.
\end{itemize}
There are many reasons why developers port code from one architecture to another. These reasons
include performance enhancements, deprecation of old hardware or trying to make use of features
that are available only on a specific architecture. Frakes et al \cite{frakes1995sixteen} discovered
that the persons involved in the developing process prefer to reuse code instead of build it from
scratch. This happens because the effort of porting an existing piece of code is often lower than
the effort of writing it from the ground up. This is true especially in the enterprise environment
where people try to patch different systems together in order to have a fast deliverable
\cite{rettig2007trouble}.

Software development costs and invested resources are big. We would like to preserve these
investments when the need for a new architecture arises. To achieve this goal people designed
programming languages and compilers that increased portability, operating systems that could run
on multiple hardware platforms \cite{johnson1978unix} and various standards that allowed developers
to talk to the computer using well defined interfaces \cite{walli1995posix}. However the porting
process is a non-trivial endeavor up to this day, it is error prone and badly formalized. The
happens because the initial architecture contains implicit assumptions about the environment as:
endianess, non-standard compiler behavior or custom changes made in the operating system structure;
or it may happen because the developers make use of non-portable constructs as ifdefs
\cite{spencer1992ifdef} or machine dependant code. Moreover these parts of the design and
environment are recorded in human-readable documents that are hard to preserve over time or the
information is not recorded at all. Thus forcing the developer to rely on its own experience and
skills in order to solve the porting problems that will arise.

Ghandorh et al \cite{ghandorh2020systematic} discovered that the porting process relies on ad-hoc
techniques. Their work presents an unified testing framework for assuring the quality of porting
a system from one architecture to another. However this does not solve the whole problem as there
is no standardized process of describing the work of porting per se. Each system has its own
peculiarities as discussed above, making the standardization process hard to formulate. In this
paper we try to propose such a standard in order to minimize the efforts and increase the
productivity of the developers.

To succesfully complete this task, we describe the experience of porting the existing Ixia testing
infrastructure, IxOS, on ARM devices. Currently it is implemented for x86, MIPS and PowerPC. The
porting involved multiple stages of planning and execution because the system was heavily rooted
in its ecosystem, making it hard for us to decouple the desired components. From build system to
interaction with the other components, IxOS proved itself hard to port because problems and errors
arose out of nowhere. For example the compilation toolchain had missing macro definition that needed
to be manually solved, bugs appeared in system libraries because we compiled the code on a new
hardware architecture and specific zones of logic needed to be hardcoded in order to keep
compatibility with the ecosystem. However we successfully ported the code on the ARM device of our
choice, Raspberry Pi 4.

To evaluate our porting results we chose three types of testing:
\begin{itemize}
	\item functional testing. We assured using manual testing that the main features of the
		system were well preserved on the new architecture.
	\item porting testing. The quality of the porting process was measured using techniques
		described in Ghandorh et al \cite{ghandorh2020systematic}. Some of these techniques
		include measurements as environment independecnce, degree of portability or porting
		costs.
	\item performance testing. Using this method we tried to understand if the system behaves
		better or werse than the old architecture in terms of memory consumption, cpu load
		and total time spend in intensive areas.
\end{itemize}

In this work we make the following contributions. First, we conduct a comprehensive study on the
porting process of systems to various architectures, identifying common difficulties,
inconsistencies and best practices. Second, we expose the experience of porting a large scale
enterprise project as IxOS on an ARM off-the-shelf board, Raspberry Pi. By doing this we focus on
the difference between accidental and essential tasks as described by Brooks et al
\cite{brooks1987no}, i.e. the difference between porting and portability. Third, we propose a
standardized recipe for porting a system to a new architecture and discuss about the requirements
for obtaining maximum efficiency for this recipe. The discussion inclueds topics as the
importance of documentation, efficient methods for documenting ideas and system architecture designs
that maximize portability.

The rest of this work is organized as follows. Section 2 and 3 discuss the background and related
work with regards to portability. Section 4 discusses the previous work done inside Keysight that
facilitated the completion of our porting experiment. Section 5 described the porting of IxOS
infrastructure on Raspberry Pi. Section 6 explores how the porting process and porting results were
evaluated. Section 7 gives a retrospective on our work and formalized the porting process. Section
8 presents a conclusion to this work and enumerates the future work.

\section{Context}
O scurtă introducere a proiectului, motivație, explicație de ce este relevant domeniul proiectului.
\section{Problema}
Care este problema pe care proiectul o va rezolva.
\section{Obiective}
Care sunt obiectivele proiectului/soluției/abordării/ideii; Ce creșteri sau evoluții determină rezolvarea proiectului.
\section{Soluția propusă}
Descrierea pe scurt a soluției implementate; ce abordare este propusă (nu detalierea utilitarelor și a tehnologiilor, ci abordarea și ideea propusă de către autor).
\section{Rezultatele obținute}
Descriere pe scurt a rezultatelor obținute, eventual de ce acestea sunt importante față de alte soluții sau studii.
\section{Structura lucrării}
Un paragraf în care fiecare dintre secțiunile următoare este prezentată în 1-2 fraze, punând accentul pe elementele cele mai semnificative din fiecare secțiune.



\chapter{Background and Related Work}

A software process is a sequence of activities that lead to the production of a software system
when executed \cite{humphrey1995discipline}. These processes can be observed in all kinds of
software development settings, allowing people who use them to have a well defined set of
expectations and results after implementing the respective process. Porting can be regarded as a
software process, making possible for the developers to decide if the practice of porting a software
system can be improved by investing more resources as human power or automated software analysis
tools, or by making modifications to the current implementation, for example choosing to refactor
software units to make porting more manageable.

Porting is the act of producing an executable version of a software unit or system in a new
environment based on an existing version \cite{mooney1993issues}. This is seldom an easy task
because it involves a good amount of code rafactoring and architectural redesign. It can be
avoided, however, if the original design has portability incorporated by using constructs as
multi-platform libraries, modular code or standard compiler behaviors.

Mooney \cite{mooney1993issues} presents two components of the porting process: transporation and
adaptation. The first is described as the act of moving the system (code or binary executable) to a
new environment and the latter is described as the act of modifying the system in order to be
compatible with the new environment. Transportation is facilitated by communication channels
to the target environment, either online (file sharing systems, remote connections) or physical
(using data storage devices). If transportation can be neglected, adaptation is more developement
resources consuming because it implies translating the source code to the new architecture, solving
possible inconsistencies and making sure that the software system behaves in a well defined manner
when ran in the new environment.

The environment is defined as the
set of software and hardware elements that interact with the system. This includes, but it is not
restricted to: operating system, communication methods, configuration files and system variables,
hardware architecture or human interaction. Two software environments involved in the porting
process will never be the same because of the software and hardware inconsistencies. On one hand,
programs between operating systems will not work, even if the hardware is the same. For example MacOS
uses MACH for executable files while Linux uses ELF, moreover even if they follow the POSIX
standard, they may have implementation details that do not align with each other. On the other
hand, processor architectures vary from one another in the way they understand machine language
and even if they do not vary, custom hardware attached to these processors makes porting difficult as
the software must be rewritten in order to accomodate the new periferials.

Portability, on the other hand, referes to the ability of software to be ported to another
environment. In the real world, a software unit or system can never be fully portable because the
environments have their own peculiarities. Even though standards help in situations as this one,
they cannot solve the whole problem of zero-cost portability. This happens because each standard
focuses on specific requirements that the standardized system must meet. For example POSIX
(ISO/IEC 9945-1) was designed to ease the task of cross-platform software development by
establishing a set of guidelines for operating system vendors to follow. On the other hand CTRON
standard \cite{wasano1989application} was designed for common application to network nodes, whether
for switching, communications
processing, or information processing, and is aimed at improved portability through realtime
performance and fault tolerance. The two standards clearly share a common ground, namely portability,
but also differ in subtle manners (CTRON is designed for network communicaton, while POSIX is more
generic). An universal standard that combines POSIX and CTRON would indeed create a more approachable
solution for the zero-cost portability problem but would limit the intrinsic expressiveness of
each standard. Because of these differences between environments, it is more accurate to characterize
software through a degree of porability, instead of simply saying that a software system is either
portable or not.

\chapter{Studiu de Piață / Metode Existente}
\dezvoltare Ce soluții similare există pe piață? Care sunt limitările lor / pentru ce cazuri de utilizare sau pentru ce tip de clienți produsele existente pe piață nu răspund cerințelor? Care sunt indicatorii pe baza cărora sunt evaluate aceste produse, de către potențiali clienți, și unde sunt lipsurile/ care este oportunitatea generată de lipsurile acestea?

\cercetare Metode existente (sau ``State of the Art'') se referă, de regulă, la nivelul curent de dezvoltare: care este starea curentă a domeniului, unde ne găsim, care este contextul. Care sunt soluțiile actuale prezente în literatura de specialitate și care sunt limitările lor? Ce direcții de explorare sunt recomandate în literatura de specialitate? Literatura de specialitate se refera la articole științifice recente, publicate în reviste cu factor de impact mare, sau în volumele unor conferințe de top, sau în cărți.

\ambele În încheierea acestui capitol se dorește descrierea tehnologiilor folosite în lucrare, cu alternative și cu argumente convingătoare calitative și cantitative.

Criterii pentru calificativul \textit{Ne\textit{Satisfăcător}}:
\begin{itemize}
	\item \dezvoltare sunt analizate superficial câteva produse de pe piață;
	\item \cercetare analiza literaturii limitata la grupuri de cercetare din românia;
	\item \ambele sunt descrise tehnologiile folosite în lucrare.
\end{itemize}

Criterii pentru calificativul \textit{Satisfăcător}:
\begin{itemize}
	\item \dezvoltare Există un interviu, un client, analiza cerințelor este elaborată pe baza interviului.
	\item \cercetare analiza literaturii de specialitate din lume, fără poziționarea precisă a lucrării în peisajului domeniului studiat;
	\item \ambele Sunt descrise câteva tehnologii alternative pentru fiecare din tehnologiile folosite în lucrare. Există o argumentare referitoare la alegere.
\end{itemize}

Criterii pentru calificativul \textit{Bine}:
\begin{itemize}
	\item \dezvoltare Proces iterativ pe baza unor interviuri cu mai mulți clienți, dezvoltare MVP, reevaluare cerințe;
	\item \cercetare analiza literaturii de specialitate din lume, cu poziționarea precisă a lucrării în peisajul actual al domeniului studiat;
	\item \ambele Sunt descrise tehnologii alternative. Sunt analizate cantitativ și calitativ, folosite benchmarkuri și teste efectuate de student. Analiza este rezumată prin tabele și grafice.
\end{itemize}

\section{Indicații formatare figuri}

Figurile utilizate în document vor fi centrate și numerotate (de exemplu Figura~\ref{fig:pic1}).
Orice figură ce nu este realizată de către autorul lucrării va fi în mod obligatoriu citată fie la final (de exemplu Figura ~\ref{fig:pic2} este preluată din documentul \cite{}), fie cel puțin într-o notă de subsol (a se vedea Figura~\ref{fig:pic2}). Orice figură ce depășește ca dimensiune 50\% dintr-o pagină, va fi mutată la anexe. Toate figurile din cadrul tezei vor fi referite în text. Exemplu: Figura~\ref{fig:pic1} prezintă o schemă de principiu pentru un amplificator inversor cu AO.

\begin{figure}[th]
\centering
\includegraphics{pics/Pic1.png}
  \caption{Amplificator inversor}
  \label{fig:pic1}
\end{figure}

\newpage

\begin{figure}[th]
\centering
\includegraphics{pics/Pic2.png}
  \caption[Amplificator de instrumentație cu 3 AO-uri]{Amplificator de instrumentație cu 3 AO-uri\protect\footnotemark}
  \label{fig:pic2}
\end{figure}
\footnotetext{© http://www.ece.tamu.edu/sspalermo/ecen3205/Secton\%201III.pdf}

\chapter{Soluția Propusă}
Capitolul conține o privire de ansamblu a soluției ce rezolvă problema, prin prezentarea structurii / arhitecturii acesteia. În funcție de tipul lucrării acest capitol poate conține diagrame (clase, distribuție, workflow, entitate-relație), demonstrații de corectitudine pentru algoritmii propuși de autor, abordări teoretice (modelare matematică), structura hardware, arhitectura aplicației.


Criterii pentru calificativul \textit{Ne\textit{Satisfăcător}}:
\begin{itemize}
	\item	Descriere în limbaj natural.
\end{itemize}

Criterii pentru calificativul \textit{Satisfăcător}:
\begin{itemize}
	\item	Descriere + diagrame de baze de date, workflow, clase, algoritmi.
\end{itemize}

Criterii pentru calificativul \textit{Bine}:
\begin{itemize}
	\item 	Descriere + diagrame de baze de date, workflow, clase, algoritmi + descrierea unui proces prin care s-a realizat arhitectura/structura soluției.
\end{itemize}

\section{Indicații formatare formule}
Formulele matematice utilizate în document vor fi centrate în pagină și numerotate.

\begin{equation}
(x+a)^n = \sum_{k=0}^{n}\left(\begin{array}{c}n\\k\\\end{array}\right)x^ka^{n-k}
\end{equation}

\begin{equation}
f(x) = a_0 + \sum_{n=1}^{\infty}\left(a_n \cos\frac{n\pi x}{L} + b_n\sin\frac{n\pi x}{L}\right)
\end{equation}



\chapter{Detalii de implementare}
În plus fata de capitolul precedent acesta conține elemente specifice ale rezolvării problemei care au presupus dificultăți deosebite din punct de vedere tehnic. Pot fi incluse configurații, secvențe de cod, pseudo-cod, implementări ale unor algoritmi, analize ale unor date, scripturi de testare. De asemenea, poate fi detaliat modul în care au fost utilizate tehnologiile introduse in capitolul 3.


Criterii pentru calificativul \textit{Ne\textit{Satisfăcător}}:
\begin{itemize}
	\item	Sunt prezentate pe scurt scheme și pseudo-cod.
\end{itemize}
Criterii pentru calificativul \textit{Satisfăcător}:
\begin{itemize}
	\item	Descriere sumara a implementării, prezentarea unor secvențe nerelevante de cod, scheme, etc.
\end{itemize}
Criterii pentru calificativul \textit{Bine}:
\begin{itemize}
	\item	Descrierea detaliată a algoritmilor/structurilor utilizați; Prezentarea etapizată a dezvoltării, inclusiv cu dificultăți de implementare întâmpinate, soluții descoperite; (dacă este cazul) demonstrarea corectitudinii algoritmilor utilizați.
\end{itemize}

\section{Indicații formatare tabele}
Se recomandă utilizarea tabelelor de forma celui de mai jos.  Font size :  9.
Orice tabel prezent în teză va fi referit în text; exemplu: a se vedea Tabel~\ref{tab:criterii}.

\begin{table}[th]\small\linespread{1}
\caption{Sumarizare criterii}
\label{tab:criterii}
\begin{tabular}{l >{\raggedright\arraybackslash}p{8cm} >{\raggedright\arraybackslash}p{4cm}}
\textbf{Calificativ} & \textbf{Criteriu} & \textbf{Observații} \\\hline
\textbf{Nesatisfacator} & Sunt prezentate pe scurt scheme și pseudo-cod & \\\hline
\textbf{Satisfacator} &Descriere sumara a implementării, prezentarea unor secvențe nerelevante de cod, scheme, etc.& \\
\hline
\textbf{\textit{Bine}} &Descrierea detaliată a algoritmilor/structurilor utilizați; Prezentarea etapizată a dezvoltării, inclusiv cu dificultăți de implementare întâmpinate, soluții descoperite; (dacă este cazul) demonstrarea corectitudinii algoritmilor utilizați. & Pot fi incluse configurații, secvente de cod, pseudo-cod, implementări ale unor algoritmi, analize ale unor date, scripturi de testare. \\
\hline
\end{tabular}
\end{table}


\chapter{Evaluare}
Acest capitol trebuie să răspundă, în principiu, la 2 întrebări și să se încheie cu o discuție a rezultatelor obținute. Cele doua întrebări la care trebuie sa se răspundă sunt:
\begin{enumerate}
	\item  \textbf{Merge corect?} (Conform specificațiilor extrase în capitolul 2);
Evaluarea dacă merge corect se face pe baza cerințelor identificate în capitolele anterioare.

	\item Cât de \textit{Bine} merge / cum se compară cu soluțiile existente? (pe baza unor metrici clare).
Evaluarea cât de \textit{Bine} merge trebuie să fie bazată pe procente, timpi, cantitate, numere, \textbf{comparativ cu soluțiile prezentate în capitolul 3}. Poate fi vorba de performanță, overhead, resurse consumate, scalabilitate etc.
\end{enumerate}

În realizarea discuției, se vor utiliza tabele cu procente, rezultate numerice și grafice. În mod obișnuit, aici se fac comparații și teste comparative cu alte proiecte similare (dacă există) și se extrag puncte tari și puncte slabe. Se ține cont de avantajele menționate și se demonstrează viabilitatea abordării / aplicației, de dorit prin comparație cu alte abordări (dacă acest lucru este posibil). Cuvântul cheie la evaluare este ``metrică'': trebuie să aveți noțiuni măsurabile și cuantificabile. În cadrul procesului de evaluare, explicați datele, tabelele și graficele pe care le prezentați și insistați pe relevanța lor, în următorul stil: ``este de preferat ... deoarece …''; explicați cititorului nu doar datele ci și semnificația lor și cum sunt acestea interpretate. Din această interpretare trebuie să rezulte poziționarea proiectului vostru printre alternativele existente, precum și cum poate fi acesta îmbunătățit în continuare.

Criterii pentru calificativul \textit{Ne\textit{Satisfăcător}}:
\begin{itemize}
	\item Aplicația este testată dar rulează pe calculatorul studentului, nu există posibilități de testare, nu a fost validată cu clienți / utilizatori;
	\item Nu au fost realizate comparații cu alte sisteme similare.
\end{itemize}

Criterii pentru calificativul \textit{Satisfăcător}:
\begin{itemize}
	\item \dezvoltare  Există teste unitare și de integrare, există o strategie de punere în funcțiune (deployment), există validare minimală cu clienții / utilizatorii.
	\item \cercetare Principalele componente și soluția în ansamblu au fost evaluate din punct de vedere al performanței, însă nu sunt folosite seturi de date standard, există unele erori de interpretare a datelor.
	\item \ambele Discuție minimală asupra relevanței rezultatelor prezentate, comparație minimală cu alte sisteme similare.
\end{itemize}

Criterii pentru calificativul \textit{Bine}:
\begin{itemize}
	\item \dezvoltare Teste unitare și de integrare, instrumente de punere in funcțiune (deployment) utilizate și care arată lucru constant de-a lungul semestrului, lucrare validată cu clienții / utilizatorii, produs în producție.
	\item \cercetare Componentele și soluția în ansamblu au fost evaluate din punct de vedere al performanței, folosind seturi de date standard și cu o interpretare corectă a rezultatelor.
	\item \ambele Discuție cu prezentarea calitativă și cantitativă a rezultatelor, precum și a relevanței acestor rezultate printr-o comparație complexă cu alte sisteme similare.
\end{itemize}

\chapter{Concluzii}
În acest capitol este sumarizat întreg proiectul, de la obiective, la implementare, si la relevanta rezultatelor obținute. În finalul capitolului poate exista o subsecțiune de ``Dezvoltări ulterioare''.

Criterii pentru calificativul \textit{Ne\textit{Satisfăcător}}:
\begin{itemize}
	\item	Concluziile nu sunt corelate cu conținutul lucrării;
\end{itemize}

Criterii pentru calificativul \textit{Satisfăcător}:
\begin{itemize}
	\item	Concluziile sunt corelate cu conținutul lucrării, însă nu se oferă o imagine asupra calității și relevantei rezultatelor obținute;
\end{itemize}

Criterii pentru calificativul \textit{Bine}:
\begin{itemize}
	\item	Concluziile sunt corelate cu conținutul lucrării, și se oferă o imagine precisa asupra relevantei și calității rezultatelor obținute în cadrul proiectului.
\end{itemize}

\chapter*{Bibliografie}\addcontentsline{toc}{chapter}{Bibliografie}
% * <marios.choudary@gmail.com> 2018-02-28T12:07:48.730Z:
%
% > BIBLIOGRAFIE
% Am adaugat un paragraf cu cateva detalii despre folosirea citarilor bibliografice in Latex, despre folosirea lui "\cite" si despre posibilitatea folosirii bibliografiei si direct in fisierul Latex.
%
% ^.

\begin{itemize}
	\item 	NU utilizați referințe la Wikipedia sau alte surse fără autor asumat.
	\item 	Pentru referințe la articole relevante accesibile în web (descrise prin URL) se va nota la bibliografie și data accesării.
	\item 	Mai multe detalii despre citarea referințelor din internet se pot regăsi la:
	\begin{itemize}
		\item	\url{http://www.writinghelp-central.com/apa-citation-internet.html}
		\item	\url{http://www.webliminal.com/search/search-web13.html}
	\end{itemize}
	\item 	Note de subsol se utilizează dacă referiți un link mai puțin semnificativ o singură dată; Dacă nota este citată de mai multe ori, atunci utilizați o referință bibliografică.
	\item 	Dacă o imagine este introdusă în text și nu este realizată de către autorul lucrării, trebuie citată sursa ei (ca notă de subsol sau referință - este de preferat utilizarea unei note de subsol).
	\item 	Referințele se pun direct legate de text (de exemplu ``KVM [1] uses'', ``as stated by Popescu and Ionescu [12]'', etc.). Nu este recomandat să folosiți formulări de tipul ``[1] uses'', ``as stated in [12]'', ``as described in [11]'' etc..
	\item 	Afirmațiile de forma ``are numerous'', ``have grown exponentially'', ``are among the most used'', ``are an important topic'' trebuie să fie acoperite cu citări, date concrete si analize comparative.
	\begin{itemize}
		\item	Mai ales în capitolele de introducere, ``state of the art'', ``related work'' sau ``background'' trebuie să vă argumentați afirmațiile prin citări. Fiți autocritici și gândiți-vă dacă afirmațiile au nevoie de citări, chiar și cele pe care le considerați evidente.
		\item	Cea mai mare parte dintre citări vor fi în capitolele de introducere ``state of the art'', ``related work'' sau ``background''.
	\end{itemize}
	\item 	Toate intrările bibliografice trebuie citate în text. Nu le adăugați pur și simplu la final.
	\item 	Nu copiați sau traduceți niciodată din surse de informație de orice tip (online, offline, cărți, etc.). Dacă totuși doriți să oferiți, prin excepție, un citat celebru - de maxim 1 frază- utilizați ghilimele și evident menționați sursa. .
	\item 	Dacă reformulați idei sau creați un paragraf rezumat al unor idei folosind cuvintele voastre, precizați cu citare (referință bibliografică) sau cu notă de subsol sursa sau sursele de unde ați preluat ideile.
\end{itemize}

Trebuie respectat un singur standard de trimiteri bibliografice (citare), dintre următoarele alternative:
\begin{itemize}
	\item APA (\url{http://pitt.libguides.com/c.php?g=12108\&p=64730})
	\item IEEE (\url{https://ieee-dataport.org/sites/default/files/analysis/27/IEEE\%20Citation\%20Guidelines.pdf})
	\item Harvard (\url{https://libweb.anglia.ac.uk/referencing/harvard.htm})
	\item Cu numerotarea referințelor în ordine alfabetică sau în ordinea apariției în text (de exemplu, stilul cu numere folosit de unele publicații ACM - \url{https://www.acm.org/publications/authors/reference-formatting})
\end{itemize}

În Latex este foarte ușor să folosiți referințe într-un mod corect și unitar, fie prin adăugarea unei secțiuni
\verb!\begin{thebibliography}!
(vezi la sfârșitul acestei secțiuni), fie printr-un fișier separat de tip bib, folosind comanda
\verb!\bibliography{}!,
așa cum procedăm mai jos prin folosirea fișierului ``bibliography.bib''. În orice caz, în Latex va trebui să folosiți comanda
\verb!\cite{}!
pentru a adăuga referințe, iar această comandă trebuie folosită direct în text, acolo unde vreți sa apară citația, ca în exemplele următoare:
\begin{itemize}
	\item Articol jurnal: ~\cite{article};
	\item Articol conferință:~\cite{proc};
	\item Carte: ~\cite{book};
	\item Weblink: ~\cite{silva};
\end{itemize}

\textbf{Important}: în această secțiune de obicei apar doar intrările bibliografice (adică doar listarea referințelor). Citarea lor prin comanda cite și explicații legate de ele trebuie facute în secțiunile anterioare. Citarea de mai sus a fost facută aici doar pentru exemplificare.

% Asa se specifica folosirea unui fisier cu referinte bibliografice:
\bibliographystyle{plain}
\bibliography{bibliography}

%% O alta varianta ar fi fost includerea de articole direct in acest fisier
%% in felul urmator:
%% \begin{thebibliography}{ABC}
%%
%% \bibitem{article}
%%  H. Baali, H. Djelouat, A. Amira and F. Bensaali,
%%  ``Empowering Technology Enabled Care Using IoT and Smart Devices:
%   A Review''. In: IEEE Sensors Journal, vol. 322 (10), pp. 891--921, 1905.
%%
%% (more \bibitem items here...)
%%
%% \end{thebibliography}

%% Daca vreti ca o sectiune sa inceapa pe o pagina noua, puteti forta acest lucru cu comanda "\newpage", ca mai jos:

%\newpage

\chapter*{Anexe}\addcontentsline{toc}{chapter}{Anexe}

Anexele sunt opționale.
Ce poate intra în anexe:
\begin{itemize}
\item	Exemplu de fișier de configurare sau compilare;
\item	Un tabel mai mare de o jumătate pagină;
\item	O figura mai mare mai mare de jumătate pagină;
\item	O secvență de cod sursa mai mare de jumătate pagină;
\item	Un set de capturi de ecran (``screenshot''-uri);
\item	Un exemplu de rulare a unor comenzi plus rezultatul (``output''-ul) acestora;
\item 	În anexe intră lucruri care ocupă mai mult de o pagină ce ar întrerupe firul natural de parcurgere al textului.
\end{itemize}

\begin{appendices}

\chapter{Extrase de cod} % Introduce o nouă anexă
\ldots


\end{appendices}
\end{document}
