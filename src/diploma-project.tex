\documentclass[12pt,a4paper]{report}

\usepackage[utf8]{inputenc} % pentru suport diacritice
%\usepackage[romanian]{babel} % setări pentru limba română
\renewcommand\familydefault{\sfdefault} % sans serif

\usepackage[margin=2.54cm]{geometry}	% dimensiuni pagină și margini
\usepackage{graphicx} % support the \includegraphics command and options

% formatting sections and subsections
\usepackage{textcase}
\usepackage[titletoc, title]{appendix}
\usepackage{titlesec}
\titleformat{\chapter}{\large\bfseries\MakeUppercase}{\thechapter}{2ex}{}[\vspace*{-1.5cm}]
\titleformat*{\section}{\large\bfseries}
\titleformat*{\subsection}{\large\bfseries}
\titleformat*{\subsubsection}{\large\bfseries}

\usepackage{chngcntr}
\counterwithout{figure}{chapter} % no chapter number in figure labels
\counterwithout{table}{chapter} % no chapter number in table labels
\counterwithout{equation}{chapter} % no chapter number in equation labels

\usepackage{booktabs} % for much better looking tables
\usepackage{url} % Useful for inserting web links nicely
\usepackage[bookmarks,unicode,hidelinks]{hyperref}

\usepackage{array} % for better arrays (eg matrices) in maths
\usepackage{paralist} % very flexible & customisable lists (eg. enumerate/itemize, etc.)
\usepackage{verbatim} % adds environment for commenting out blocks of text & for better verbatim
\usepackage{subfig} % make it possible to include more than one captioned figure/table in a single float
\usepackage{enumitem}
\setlist{noitemsep}

\usepackage{etoolbox}
\usepackage{setspace}
\usepackage{sectsty}
\usepackage{titlesec}
\usepackage{graphicx}
\usepackage{indentfirst}
\usepackage{authblk}
\usepackage{xcolor}
\usepackage{calc}
\usepackage{multirow}
\usepackage{tikz}
\usepackage{cite}

\usepackage{fancyhdr}

\pagestyle{empty}
\renewcommand{\headrulewidth}{0pt}
\renewcommand{\footrulewidth}{0pt}
\lhead{}\chead{}\rhead{}
\lfoot{}\cfoot{\thepage}\rfoot{}

\newcommand{\HeaderLineSpace}{-0.25cm}
\newcommand{\UniTextRO}{UNIVERSITATEA POLITEHNICA DIN BUCUREȘTI \\[\HeaderLineSpace]
FACULTATEA DE AUTOMATICĂ ȘI CALCULATOARE \\[\HeaderLineSpace]
DEPARTAMENTUL DE CALCULATOARE\\}
\newcommand{\DiplomaRO}{PROIECT DE DIPLOMĂ}
\newcommand{\AdvisorRO}{Coordonatori științifici:}
\newcommand{\BucRO}{BUCUREȘTI}

\newcommand{\UniTextEN}{UNIVERSITY POLITEHNICA OF BUCHAREST \\[\HeaderLineSpace]
FACULTY OF AUTOMATIC CONTROL AND COMPUTERS \\[\HeaderLineSpace]
COMPUTER SCIENCE AND ENGINEERING DEPARTMENT\\}
\newcommand{\DiplomaEN}{DIPLOMA PROJECT}
\newcommand{\AdvisorEN}{Thesis advisors:}
\newcommand{\BucEN}{BUCHAREST}

\newcommand{\frontPage}[6]{
\begin{titlepage}
\begin{center}
{\Large #1}  % header (university, faculty, department)
\vspace{50pt}
\begin{tabular}{p{6cm}p{4cm}}
\includegraphics[scale=0.8]{pics/upb-logo.jpg} &
	\includegraphics[scale=0.5,trim={14cm 11cm 2cm 5cm},clip=true]{pics/cs-logo.pdf}
\end{tabular}

\vspace{105pt}
{\Huge #2}\\                           % diploma project text
\vspace{40pt}
{\Large #3}\\ \vspace{0pt}  % project title
{\Large #4}\\                          % project subtitle
\vspace{40pt}
{\LARGE \Name}\\                   % student name
\end{center}
\vspace{60pt}
\begin{tabular*}{\textwidth}{@{\extracolsep{\fill}}p{6cm}r}
&{\large\textbf{#5}}\vspace{10pt}\\      % scientific advisor
&{\large \Advisor}\\                     % first advisor
&{\large \AdvisorTwo}                    % second advisor
\end{tabular*}
\vspace{20pt}
\begin{center}
{\large\textbf{#6}}\\                                % bucharest
\vspace{0pt}
{\normalsize \Year}
\end{center}
\end{titlepage}
}

\newcommand{\frontPageRO}{\frontPage{\UniTextRO}{\DiplomaRO}{\ProjectTitleRO}{\ProjectSubtitleRO}{\AdvisorRO}{\BucRO}}
\newcommand{\frontPageEN}{\frontPage{\UniTextEN}{\DiplomaEN}{\ProjectTitleEN}{\ProjectSubtitleEN}{\AdvisorEN}{\BucEN}}

\linespread{1.15}
\setlength\parindent{0pt}
\setlength\parskip{.28cm}

%% Abstract macro
\newcommand{\AbstractPage}{
\begin{titlepage}
\textbf{\large SINOPSIS}\par
\AbstractRO\par\vfill
\textbf{\large ABSTRACT}\par
\AbstractEN \vfill
\end{titlepage}
}

%% Thank you macro
\newcommand{\ThanksPage}{
\begin{titlepage}
{\noindent \large\textbf{MULȚUMIRI}}\\
\Thanks
\end{titlepage}
}

%%% Puteți elimina aceste linii din lucrare, servesc numai pentru template.
\newcommand{\worktype}[1]{[\textit{#1}] }
\newcommand{\dezvoltare}{\worktype{Dezvoltare de produs}}
\newcommand{\cercetare}{\worktype{Cercetare}}
\newcommand{\ambele}{\worktype{Ambele}}
%%%

%%
%%   Campurile de mai jos trebuie modificate de autor. Modificati doar continutul, nu si numele fiecarei definitii
%%
\newcommand{\ProjectTitleRO}{Studiu de Caz al Evaluarii Costurilor de Portare:
Portarea IxOS pe placute ARM \todo{cred ca ar trebui sa aiba diacritice}}
\newcommand{\ProjectSubtitleRO}{~ }
\newcommand{\ProjectTitleEN}{Case Study of Evaluating Porting Costs: Porting IxOS on ARM Boards}
\newcommand{\ProjectSubtitleEN}{~ }
\newcommand{\Name}{Lucian-Ioan Popescu}
\newcommand{\Advisor}{Dr. ing. Lucian Mogosanu}
\newcommand{\AdvisorTwo}{Conf. Dr. ing. Razvan Deaconescu}
\newcommand{\Year}{2022}

\newcommand\todo[1]{\textcolor{red}{#1}}
\newcommand{\lm}[1]{\textcolor{blue}{\bf [LM: #1]}}
\newcommand{\lp}[1]{\textcolor{red}{\bf [LP: #1]}}


% Setări document
\title{Proiect de diplomă}
\author{\Name}
\date{\Year}

\begin{abstract}

Software porting is costly and time consuming. As developers prefer to reuse
code instead of rewriting it, it is important to understand the framework for
quantifying the costs so that the process can be optimized based on these exact
costs. In this paper we port the Ixia network testing infrastructure on ARM
boards, such as Raspberry Pi, and extract the porting costs associated with this
process. By doing this we create a revised model of porting based on which we
extract the costs for our project. Furthermore we present the factors that
affect the costs and their direct correspondence with the porting model and
costs. Finally, we discuss the limitations of our porting model and make a
comparison with the old porting model.

\end{abstract}

\begin{keywords}
Network testing, porting costs, porting model, porting factors
\end{keywords}


%%
%%   Campurile aferente paginii de multumiri
%%
\newcommand{\Thanks}{(opțional) Aici puteți introduce o secțiunea specială de mulțumiri / acknowledgments. }

\begin{document}

\frontPageRO
\frontPageEN

\begingroup
\linespread{1}
\tableofcontents
\endgroup

\AbstractPage
\ThanksPage

\section{Introduction}

Porting a software system and evaluating the porting costs is a hard problem in
systems programming.~\cite{b1,b2,b4,b5,b9,b10,b11,b12,b13,b14,b15,b16} The
reasons for porting software systems are various: the developers want to enhance
the performance, the hardware environment starts to get deprecated or the system
wants to take advantage of features unavailable in the current environment.

This paper highlights the porting issues associated with porting the IxOS
testing infrastructure, regarding both the hardware and software components, to
ARM off-the-shelf boards, Raspberry Pi 4 in our case. In the \textit{Background}
section we present different terms associated with porting: what porting is, a
porting model, porting costs and factors. In the \textit{Porting IxOS on ARM
boards} we divide our work in multiple steps. For each step we present a
description, the targets and milestones, and the final results.  After the
porting process is discussed, we highlight the \textit{Porting Costs Evaluation}
associated with our work, including an analysis of the factors that influenced
the presented costs.  Finally, in \textit{Discussions on Porting Costs} we
conduct a discussion about the porting process and the porting costs where we
investigate the porting difficulties, observations about the porting tasks and
ways to improve the porting process. 

\chapter{The anatomy of the porting process} \label{sec:background}

The process of reusing code in a new environment has clear advantages over
rewriting~\cite{frakes1995sixteen}. However this process does not come cheap, as
the task of reusing code through porting is time consuming. In this chapter we
present the theoretical background and the related work in the field of porting
including the porting process with its tasks, porting costs and porting factors.

\section{Porting and porting tasks}

Porting is the act of producing an executable version of a software unit or
system in a new environment based on an existing
version~\cite{mooney1990strategies}. This is seldom an easy task because, in
general, it involves a good amount of code refactoring and rewriting. It can be
avoided, however, if, in particular, the original design has portability
incorporated by using, for example, constructs as multi-platform libraries,
modular code or standard compiler behaviors~\cite{tanenbaum1978guidelines}.

The environment is defined as the set of software and hardware elements that
interact with the system. This includes, but it is not restricted to: operating
system, communication methods, configuration files and system variables,
hardware architecture or human interaction. Two software environments involved
in the porting process will never be the same because of the software and
hardware inconsistencies. On one hand, programs between operating systems will
not work, even if the hardware is the same. For example MacOS uses MACH for
executable files while Linux uses ELF, moreover even if they follow the POSIX
standard, they may have implementation details that do not align with each
other. On the other hand, processor architectures vary from one another in the
way they understand machine language and even if they do not vary, custom
hardware attached to these processors may make porting difficult as the software
must be rewritten in order to accomodate the new peripherals.

Mooney~\cite{mooney1990strategies} presents two components of the porting process:
transporation and adaptation. The first is described as the act of moving the
system (code or binary executable) to a new environment and the latter is
described as the act of modifying the system in order to be compatible with the
new environment. Transportation is facilitated by communication channels to the
target environment, either online (file sharing systems, remote connections) or
physical (using data storage devices). Adaptation consumes more development
resources than transportation because it implies translating the source code to
the new environment, solving possible inconsistencies and making sure that the
software system behaves in a well defined manner when ran in the new
environment.

Mooney's model is very simplistic regarding the tasks that can occur during a
porting process. Hakuta and Ohminami~ \cite{hakuta}, and Tanaka et
al.~\cite{tanaka} created a more accurate model that reflects better components
involved in porting an application. The tasks involved in their model are the
following:
\begin{itemize}
    \item Advance preparations
        \begin{itemize}
            \item Surveying development environment
            \item Surveying OS
            \item Surveying program for porting
            \item Surveying workstation development environment
            \item Adjusting target environment
            \item Initial source code modifications
        \end{itemize}
    \item Workstation testing
        \begin{itemize}
            \item Standalone testing on workstation
            \item Linked testing on workstation
        \end{itemize}
    \item Target testing
        \begin{itemize}
            \item File-making
            \item File system creation
            \item Installation on target
            \item Test program creation
            \item Linked test on target
        \end{itemize}
    \item General duties
        \begin{itemize}
            \item Documentation
            \item Progress tracking
            \item Discussions
        \end{itemize}
\end{itemize}

They emphasize on spending additional time on getting familiar with the system
and only then starting to port the application per se. Testing is conducted in a
workstation environment (that is, testing the builds on the local machine or in
a local simulator/emulator) and in the target environment. Finally they also
include non-technical duties as documentation, tracking and discussions.

\section{Porting costs and factors}

Porting costs, and more generally, software development costs, are measured in
man-hours~\cite{tanaka, hakuta}. While the costs are determined by program size
and contents~\cite{hakuta}, other factors as portability impediments, human
factors or environmental factors~\cite{hakuta} play a considerable role. To
understand the factors that influence the porting costs, these factors are
quantified in indices that describe how much of an influence they have.

In our work we will use three indices of this type as follows: portability
impediments index, human factors index and environmental factors index.

The first index, portability impediments, answers the following question: how
portable was the program to be ported and how many difficulties did the
developer meet in the porting process? The factors that influence this index are
described in (S1$\sim$S11~\cite{hakuta}) and the index is computed using the
following formula: \[ \alpha_p = \eta * \sum_{n=1}^{12} \omega_i S_i \]

Here $\eta$ is a portability design index, $\omega_i$ is the weight assigned to
each factor and $S_i$ is 1 when the impediment factor $i$ exists, otherwise is
0. The factors are placed in three categories: differences in processor
architecture, OS disparity and differences in language processor.

The second index, human factors, answers the following question: what role did
the experience and knowledge of the developer play in the porting process? The
factors that influence this index are described in (H1$\sim$H5~\cite{hakuta}) and
the index is computed using the following formula: \[ \sum_{n=1}^{5} H_i \]

Here $H_i$ are the human factors presented in~\cite{hakuta}. Their values range
from -2 which reflects the maximum productivity while 2 reflects the minimum
productivity.

The third index, environmental factors, answers the following question: how did
the development and testing environments, and the tools used during the porting
process affect the porting costs? The factors that influence this index are
described in (E1$\sim$E3~\cite{hakuta}) and the index is computed using the
following formula: \[ \sum_{n=1}^{3} E_i \]

Here $E_i$ are the following environmental factors as presented in~\cite{hakuta}:
\begin{itemize}
    \item Development environment (E1)
    \item Unit test environment (E2)
    \item System test environment (E3)
\end{itemize}

As for H1$\sim$H5, the values for E1$\sim$E3 range between -2 and 2, -2 being the best
score for $E_i$, while 2 being the worst.

After we described the anatomy of the porting process and presented the porting
costs associated with this process, we present a practical example of porting a
software system and evaluating the costs.

\chapter{Porting IxOS on ARM Boards} \label{sec:portingIxos}

To continue our study of evaluating the porting costs, we chose to port a large
scale system used for network testing. In this section we will present the
process of porting this system. 

\section{Porting Architecture}

The components of the porting architecture, presented in
Figure~\ref{fig:ixos_arch}, are:
\begin{itemize}
    \item A client for printing network testing data (\textbf{IxExplorer})
    \item A middleware for connecting the client to the machines that run the
    network testing suite (\textbf{Chassis})
    \item The actual machines that run the testing (\textbf{Cards})
    \item The device that benefits from network testing (\textbf{Device Under Test} - DUT)
\end{itemize}

IeExplorer is a Windows application that allows the user to connect to the
machines that run the testing, configure and run tests on them, and retrieve
information about the status of the tests.

To allow IxExplorer to easily connect to multiple machines that run the testing,
a Chassis is used. The most important application that runs on the Chassis is
IxServer. It creates a communication channel between IxServer and the testing
machines. 

Next, the cards are the most interesting part for our porting. They run on a
custom Ixia solution, IxVM to achieve the best performance for network testing. On top
of IxVM is placed IxOS Linux, a modified version of Linux with custom
device drivers, kernel parameters and userspace applications. The two most
interesting applications for our porting are InterfaceManager (IM) and IxStack.
Their role in the system is to manage the network interfaces, represented by
Port01 and Port02 in Figure~\ref{fig:ixos_arch}, and to load the network
protocols for the testing suites.

The device device under test is the consumer of the network traffic produced by
the cards. Usually it is connected between two \textbf{Ports} that will monitor the
behavior of the DUT while receiving traffic.

\begin{figure}
    \centering
    \def\svgscale{0.95}
    \input{fig/ixos_arch.pdf_tex}
    \caption{IxOS porting architecture}
    \label{fig:ixos_arch}
    \medskip
    \small
    The right part of the figure presents the modification of the card
    environment.
\end{figure}

\section{New IxOS Architecture}

In Figure~\ref{fig:ixos_arch} we highlighted the changes we made to the initial
architecture. First, we changed the hardware a card runs on. From IxVM we moved
to a Raspberry Pi 4. On top of the Raspberry Pi, we put a Raspberry Pi OS Linux
instead of IxOS Linux.

These changes required us to port InterfaceManager and IxStack to the new
Raspberry Pi environment so that we could benefit from the same functionality as in the
old Ixia custom hardware + IxOS Linux environment. The process of porting the
two applications is described in the next section.

To recreate the required environment for IM and IxServer, we had to install
custom tools, named pipes and shared libraries in the target environment.

\section{Previous work in this area}

Before we started the project, there existed an attempt to make InterfaceManager
and other applications independent of IxOS. The previous project focused on
compiling the whole IxOS infrastructure for x86 and then extract the binaries
for InterfaceManager and other relevant applications.

This helped us during our porting because the result of the previous work was a
portable application that could easily be moved in another Linux environment.
Making InterfaceManager independent of IxOS meant that all the assumptions made
by the application regarding the operating system interface were removed (e.g.,
custom device drivers and proc entries) and the work of porting it to another
environment was simply a task of finding the right tools for building the
binaries and solving unknown inconsistencies.

InterfaceManager had already a high degree of portability as it was written
in C++ using an object oriented paradigm. The architecture dependent code was
separated using constructs as ifdefs and the coding style was compiler agnostic,
meaning that we were able at any time to plug another compiler and generate the
correct binaries.

\section{The Porting Process}

The porting process consisted in moving the IxOS testing infrastructure from its
source environment to a new target environment consisting of an off-the-shelf
ARM board, Raspberry Pi, running a stock Linux operating system. When we started
the porting we set the following porting milestones:
\begin{itemize}
    \item Separate InterfaceManager from IxOS in a git repository
    \item Run InterfaceManager in QEMU
    \item Run InterfaceManager on Raspberry Pi hardware
\end{itemize}
We achieved all our proposed milestones during the twelve weeks of the project.
At the end of the project we divided the work into three logical stages
that cover the process of achieving the above milestones.

\subsection{Porting Stages}

The logical stages of our porting process are the following:
\begin{itemize}
    \item Build binaries for ARM64
    \begin{itemize}
        \item Separate IM + HostProxy from IxOS infrastructure
        \item Integrate IxStack with IM
        \item Build plugins for IM
        \item Test initial builds in QEMU
    \end{itemize}
    \item Create Card environment on RPi
    \begin{itemize}
        \item Run setup on RPi hardware
        \item Modify IM to run in the new environment
        \item Debug IM init time issues
    \end{itemize}
    \item Bring up ports on Chassis
    \begin{itemize}
        \item Find the cause of the IM init time issues
        \item Solve the problems regarding the link state of the card
    \end{itemize}
\end{itemize}
These stages with their respective substages were not completed in chronological
order. The process of completion was rather incremental, meaning that for
example we had to test the initial builds in QEMU each time we made a
modification in the \textit{Integrate IxStack in IM} stage, going back and forth
between the two substages.

In the first stage, we started the work of porting by trying to separate the
relevant components from IxOS and build them for ARM64. We started with
InterfaceManager and HostProxy, which were the building blocks of our porting,
and continued with IxStack and its components. The hardware was not available
at this time so we tested our builds using an emulator for the target
architecture (i.e., QEMU).

Next, when the hardware arrived, we used the Raspberry Pi to test our builds.
This proved to be a great advantage for us because the testing environment
worked better on hardware than in the emulator. After we built all our
components, we started to focus on the inconsistencies between the source
environment and the target environment. For that we had to modify some parts of
the program that were source environment dependent. However the porting did not
run as smoothly as we expected. We had to investigate multiples paths to
understand the problems we faced, namely the problems regarding the
initialization of our Card.

In stage three we investigated the initialization problems from another
perspective. This time we investigated what was the initialization sequence on
IxServer. Here we discovered that IxServer did not set the link state correctly
when connecting to our Card. To solve that we tracked the variables and code
zones that modify the link state in order to find out which code lines cause
trouble. In the end we discovered that a problem regarding the initialization
of the protocol loader was causing our start-up issues. After we solved
these issues we were able to fully port InterfaceManager on Raspberry Pi and
integrate it with the bigger system consisting of IxServer and IxExplorer.

\section{General difficulties}

\todo{this might fit better in the discussions chapter}
After describing the concrete impediments we faced during our porting we decided
to understand them and create a list with general difficulties that may apply
to other porting processes. The list is found in Table~\ref{tab:portImpeds}.

\begin{table*}
\centering
\begin{tabular}{|p{4cm}|p{10cm}|}
\hline
Difficulty & Details \\
\hline
Lacking documentation & Lacking written documentation about how the system works means
that the developer must either figure out the system alone or must communicate
with other developers in order to gather information about the system. This adds
overhead to the porting process as documentation through communication is
slower than documentation through written text.\\
\hline
Inconsistencies between environments & This difficulty corresponds to the
degree of which the application is portable~\cite{b4}. If the degree of
portability is too low (this depends entirely on the application) then the
developer will be faced with many inconsistencies between the source and target
environments that will be reflected in the cost of porting (i.e., man-hours). \\
\hline
Use of tools & Using inadequate development and testing tools introduces additional
overhead in the porting process. This happens when the used tools are either
outdated or too hard to use for the purpose of the process. \\
\hline
Understanding the system & Software complexity is a multi-dimensional problem,
it includes: structural, computational, logical, conceptual and textual complexity.
~\cite{b6} There is no easy way to understand the system, so the developer will
be faced with the task of understanding the architecture diagrams, huge code
base, written documents and tutorials either when the porting starts or during
the porting process. \\
\hline
Unexpected difficulties & As with every software process, there are difficulties
that can not be estimated beforehand. They include unknown parts of the software
that may cause problems when ported or assumptions about the environment made by
the system that will not work in the target environment. The only time they
are clearly reflected in the porting costs is at the end of the project where
a retrospective is led. \\
\hline
\end{tabular}
\caption{General porting difficulties \todo{fix this table}}
\label{tab:portImpeds}
\end{table*}

\section{Porting Costs Evaluation}

% What I want to say in this section:
%   - I want to take the model of Tanaka and map my timeline on it (maybe merge
%   the model with [1])
%   - I want to talk about the porting tasks
%     - which was most time consuming
%     - which of them were repetitive and added little value
%     - etc
%   - I want to talk about the human and development factors as described in [1]
%   - talk about environment disparity and program factors; and how did this
%   affect the costs
%   - Finally I need to understand the equations for porting costs evaluation
%   and estimation in order to compare the resluts (and maybe talk about the
%   porting productivity index)
%
% The Tanaka+Hakuta model mapped on our timeline will be presented in a table.
% The indices from Hakuta([1]) will be presented in different subsection (idk).
%
% [1] "A study of Software Portability Evaluation"

In this section we present the costs associated with the porting process
described in the previous section. First we divide our work in tasks that can
be individually evaluated and later we discuss the factors that affected our
cost evaluation.

\subsection{Man-days costs}

In order to have an accurate cost evaluation of the porting process, we divided
our process into multiple subprocesses that can easily be evaluated using the
tracking created during the porting. We use man-days to evaluate the
cost for each subprocess because the tracking was done each week rather than
each day.

The anatomy of the porting process, as described in Tanaka et al. (citation
needed here) and Hakuta et al. (citation needed, there are only two authors,
should I name both instead saying "et al."?) with modification as per our
project needs has the following structure:
\begin{itemize}
    \item Advance preparations
        \begin{itemize}
            \item Surveying development environment
            \item Surveying target OS
            \item Surveying program for porting
            \item Surveying documentation
            \item Adjusting development environment
            \item Adjusting target environment
            \item Initial source code modifications
        \end{itemize}
    \item Building for target environment
        \begin{itemize}
            \item File-making
            \item Installation on remote environment
            \item Reviewing inconsistencies between source and remote environments
            \item Solving problems with external dependencies
            \item Create testing infrastructure
        \end{itemize}
    \item Testing
        \begin{itemize}
            \item Testing in simulated environment
            \item Testing in target environment
        \end{itemize}
    \item General duties
        \begin{itemize}
            \item Documentation
            \item Progress tracking
            \item Discussions
        \end{itemize}
\end{itemize}

Here we need a paragraph that clarifies some of the porting subprocesses.

This structure assumes that porting is a linear process, which is not the case.
Many tasks are repetitive between different stages of the project. For example
file-making occurs before and after testing in target environment takes place,
making it hard to accurately track each subtask. For this reason table
~\ref{tab:manHoursEvaluation} does not track the exact hours or days spent
on a specific subtask. It rather presents in how many days a specific subtask
was finished, even if that subtask took one or two hours of that day. Finally,
the last column of the table presents the number of days to finish a subtask
relative to the total number of days.

\begin{table*}
\centering
\begin{tabular}{ |c|c|c|c|c| }
\hline
Porting task & Subtasks & Days & Relative days \\
\hline
\multirow{7}{5em}{Advance preparations} & Surveying development environment & 5 & 8.33\\
& Surveying target OS & 5 & 8.33 \\
& Surveying program for porting & 5 & 8.33 \\
& Surveying documentation & 10 & 16.66 \\
& Adjusting development environment & 10 & 16.66 \\
& Adjusting target environment & 5 & 8.33 \\
& Initial source code modifications & 0 & 0 \\
\hline
\multirow{5}{5em}{Building for target environment} & File-making & 35  & 58.33\\
& Installation on remote environment & 40 & 66.66 \\
& Reviewing inconsistencies between source and remote environments & 50 & 79.63 \\
& Solving problems with external dependencies & 5 & 8.33 \\
& Create testing infrastructure & 15 & 25.00 \\
\hline
\multirow{2}{5em}{Testing} & Testing in simulated environment & 20  & 33.33\\
& Testing in target environment & 25 & 41.66 \\
\hline
\multirow{3}{5em}{General duties} & Documentation & 15  & 25.00 \\
& Progress tracking & 12 & 20.00 \\
& Discussions & 60 & 100.00 \\
\hline
\end{tabular}
\caption{Man-days evaluation for porting tasks}
\label{tab:manHoursEvaluation}
\end{table*}

As seen from the table the most time consuming subtask are the following:
discussions, incosistencies reviewing, installation on remote environment and
file-making. The amount of time spent on these tasks is different. While
discussions and inconsistencies reviewing were planned during the day, taking
a predefined amount of time, installation on remote environment and file-making
were repetitive and sporadic tasks. However this does not decrease the
importance of the later tasks, they were time consuming because they helped us
to achieve other tasks, mainly testing tasks.

Another thing that stands out is that we spent zero days on initial source code
modifications. This happened because the code base was too unfamiliar in order
to make source code modifications from the start. The first source code
modifications were present in the reviewing inconsistencies stage because the
toolchain helped us to discover the code zones to be changed.

\subsection{Factors of porting costs}

While porting costs, in our case man-days, are determined dirctly by program size and
content, other factors and impediments as human experience and environment
disparities must be taken into consideration.

To determine the impact some of these factors had on our porting process, we
use the indices described in Hakuta and Ohminami (citation needed) that give
a quantitative influence of porting factors and impediments.

The first index that we compute is the portability impediment index, this will
show us how portable was the porgram we ported and how many difficulties did
we meet in our porting process. The index is computed using the following
formula: $\alpha_p = \eta * \sum_{n=1}^{12} \omega_i S_i$, where $\eta$ is a
portability design index, $\omega_i$ is the weight assigned to each factor and
$S_i$ is 1 when the impediment $i$ exists, otherwise is 0. In our case $eta$
has a value of 2, meaning that "the non-portable parts of the program are not
localized, but the correspondence of program codes to their functions is
clarified". For simplicity we will assume that $\omega_i$ is 0 if the impediment
was insignificant, 0.5 if the impediment had a normal difficulty and 1 if the
impediment was hard to solve. In our porting we discovered the following
portability impediments:
\begin{itemize}
    \item Difference in compiler specification (S8)
    \item Scope of library support (S9)
    \item Implementation-dependent libraries (S10)
    \item Difference in operating system interface (S12)
\end{itemize}

It can be noted that we added an additional impediment inexistent in Hakuta and
Ohminami's list, namely S12, which can be placed in the "OS disparity" category.

Given these impediments the portability impediment index has the following
value: $\alpha_p = 2 * (0.5 * S8 + 1 * S9 + 1 * S10 + 0.5 * S12) = 6$. The
maximum value for $\alpha_p$ is 36. This number is achieved because there were
no differences between processor architecture (S1~S5), little difference between
source and target OSes (S6, S7, S12 and a major difference in language processor
(S8~S11). The application was written in C++ and the port was between two Linux
versions, which highly decreased the impediments we could have faced.

\section{Discussions on Porting Costs}

% What I want to say in this section:
%   - I want to reflect on imporving the porting process we did. Tanaka proposes
%   the following seven ways of raising porting efficiency:
%     - porting guidlines
%     - porting compatibility checking tool
%     - portability evaluation tool
%     - tool for generating system calling routines
%     - program structure viewing tool
%     - os emulator
%     - test support tools
%   - I would like to see how relevant are they for our project. there is no way
%   at the moment to actually test the ways of improving porting efficiency so
%   I will keep the statements at the level of discussion, thinking about how
%   would have the process been different if we used one of the above ways (this
%   is chapter 3.2 from Tanaka)

\section{Conclusions and Further Work}

\subsection{Conclusions}
We succeeded in porting the IxOS infrastructure on ARM boards. We separated the
relevant components for our porting (i.e., InterfaceManager and IxStack) from
IxOS so that they run on every ARM-based Linux distribution. At the moment of
writing the project is in the proof-of-concept stage. If there is interest for
continuing the project or integrating it in other projects inside the company,
we provided the necessary environment for deploying it.

We have extracted the porting costs for porting IxOS infrastructure on ARM
boards. Thus we understood what the weaknesses and the strenghts of our project
were. The lack of understanding of the project structure and project use cases
proved itself to be an important factor during the process of porting. Because
of this reason we had to spend additional time on testing the system and
understanding its components that could have been used on solving problems and
inconsistencies. A strength of our project was the fact that the program for
porting had a high degree of portability. This helped us to shrink the volume of
inconsistencies between the source environment and the target environment.

We succeeded in creating a better model for software porting starting from the
model presented in~\cite{b1,b2} and from our project specific needs. We
contributed to this model by making it more generic and allowing other software
porting projects to easily map their needs on this model.

We have provided a discussion on the open problems in the area of software
porting including the non-linearity of the porting project, what implications
does this have in the extraction of porting costs and the need for a dependency
graph between the porting tasks so that progress tracking in software porting
projects could be done more efficiently. Finally, we compared our porting costs
with the costs of Tanaka et al.~\cite{b1} to understand better the differences
and patterns between our processes. 

\subsection{Further Work}

We plan to test the performance of the system so that we can achieve the other
goal we proposed in the beginning of this paper (i.e., explore Ixia testing
infrastructure on other architectures so that we can achieve a better
performance of the system). To achieve this goal we will compare our solution in
the target environment with the same solution in the source environment. We
expect to see better results in the new environment for some network testing
suites than in the old enviroment. Furthermore, it would be interesting to see
how our solution compares to other open-source solutions as {\color{red} TBD}.

To complete the analysis of the factors that affected the porting costs we plan
to analyze the characteristics of the program to be ported. We want to analyze
the program size and contents and the content of the changes needed for porting.
By doing this we hope to find a direct correspondence between the program to be
ported and specific porting subtasks (e.g., \textit{Solving inconsistencies
between source and target environment}).

Finally, we want to analyze the porting improvements guidelines presented
in~\cite{b1} and map them on our porting process. However, we do not plan to
restart the porting processs while mapping these guidelines, instead we want to
have a discussion and draw conclusions based on them.


% Asa se specifica folosirea unui fisier cu referinte bibliografice:
\bibliographystyle{plain}
\bibliography{bibliography}

%% O alta varianta ar fi fost includerea de articole direct in acest fisier
%% in felul urmator:
%% \begin{thebibliography}{ABC}
%%
%% \bibitem{article}
%%  H. Baali, H. Djelouat, A. Amira and F. Bensaali,
%%  ``Empowering Technology Enabled Care Using IoT and Smart Devices:
%   A Review''. In: IEEE Sensors Journal, vol. 322 (10), pp. 891--921, 1905.
%%
%% (more \bibitem items here...)
%%
%% \end{thebibliography}

%% Daca vreti ca o sectiune sa inceapa pe o pagina noua, puteti forta acest lucru cu comanda "\newpage", ca mai jos:

%\newpage

\chapter*{Anexe}\addcontentsline{toc}{chapter}{Anexe}

Anexele sunt opționale.
Ce poate intra în anexe:
\begin{itemize}
\item	Exemplu de fișier de configurare sau compilare;
\item	Un tabel mai mare de o jumătate pagină;
\item	O figura mai mare mai mare de jumătate pagină;
\item	O secvență de cod sursa mai mare de jumătate pagină;
\item	Un set de capturi de ecran (``screenshot''-uri);
\item	Un exemplu de rulare a unor comenzi plus rezultatul (``output''-ul) acestora;
\item 	În anexe intră lucruri care ocupă mai mult de o pagină ce ar întrerupe firul natural de parcurgere al textului.
\end{itemize}

\begin{appendices}

\chapter{Extrase de cod} % Introduce o nouă anexă
\ldots


\end{appendices}
\end{document}
