\newcommand{\AbstractRO}{
Portarea software e costisitoare si consumatoare de timp. Refolosirea software-ului a devenit o practica standard in ingineria software datorita beneficilor sale de salvare a costurilor, astfel e imortant sa intelegem cadrul de cuantificare al costurilor astfel incat procesul sa poata fi optimizat pe aceste costuri exacte. Ca sa exploram aceasta problematica, luam ca exemplu de portare portarea infrastructurii de testare Ixia pe un sistem off-the-shelf, popular, cu suport pentru Linux, ca si Raspberry Pi si extragem costurile de portare asociate cu acest proces. Acest lucru ne ajuta sa cream un model revizuit al portarii pe baza caruia extragem costurile de portare pentru proiectul nostru. Mai mult de atat, prezentam factorii care au afectat costurile si corespondenta lor directa cu modelul de portare si costurile. In sfarsit, discutam limitarile modelului nostru de portare si facem o comparatie cu modelul vechi de portare.
}

\newcommand{\AbstractEN}{
\lm{As you review the thesis, try to sum up the most interesting/relevant parts of your results and add them here, such as: what you have found about the costs of porting that was not mentioned in related work; what limitations you have found in existing work and what you did differently. This not only places your thesis in the larger context of works on software porting (as un-sexy as they are), but it also differentiates it from them.}

Software porting is costly and time consuming. Software reuse has become a
standard practice in software engineering due to its cost saving benefits,
therefore it is important to understand the framework for quantifying the costs
so that the process can be optimized based on these exact costs. In order to
explore this issue, we take our experience of porting the Ixia network testing
infrastructure on a popular, off-the-shelf system that supports Linux, such as
Raspberry Pi and extract the porting costs associated with this process. This
leads us to a revised model of porting based on which we extract the costs for
our project. Furthermore we present the factors that affect the costs and their
direct correspondence with the porting model and costs. Finally, we discuss the
limitations of our porting model and make a comparison with the old porting
model. }
