\newcommand{\AbstractRO}{

Pana in prezent portarea se bazeaza pe metode ad-hoc care au la baza factori ca experienta
dezvoltatorului sau claritatea sistemului ce se vrea portat. Acest studiu ofera o prezentare a
problemei portarii software bazandu-ne pe experienta practica de a porta infrastructura de testare Ixia
pe arhitectura ARM. Pentru acomodarea pe noua platforma am facut modificari ce tin de interfatarea
cu sistemul de operare sau de pastrarea compatibilitatii cu celelalte componente ale ecosistemului.
Testarea s-a desfasurat in mai trei etape: etapa functionala, etapa de performanta si
evaluarea procesului de portare. Rezultatele arata ca TBD. Aceste rezutlate faciliteaza formalizarea
unor tehnici de portare explicate pe larg de-a lungul lucrarii, cum ar fi impartirea si decuplarea
unitatilor sistemului sau izolarea bucatilor dependente de arhitectura.

}

\newcommand{\AbstractEN}{
Software porting is costly and time consuming. As developers prefer to reuse
code instead of rewriting it, it is important to understand the framework for
quantifying the costs so that the process can be optimized based on these exact
costs. In this paper we port the Ixia network testing infrastructure on ARM
boards, such as Raspberry Pi, and extract the porting costs associated with this
process. By doing this we create a revised model of porting based on which we
extract the costs for our project. Furthermore we present the factors that
affect the costs and their direct correspondence with the porting model and
costs. Finally, we discuss the limitations of our porting model and make a
comparison with the old porting model.
}
