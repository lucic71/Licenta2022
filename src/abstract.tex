\newcommand{\AbstractRO}{
\todo{Add Romanian translation}
}

\newcommand{\AbstractEN}{
\lm{``As developers prefer to reuse code'' -- why do they prefer this? you could go one level deeper and state that code reuse has become a standard practice due to cost saving.}

\lm{As you review the thesis, try to sum up the most interesting/relevant parts of your results and add them here, such as: what you have found about the costs of porting that was not mentioned in related work; what limitations you have found in existing work and what you did differently. This not only places your thesis in the larger context of works on software porting (as un-sexy as they are), but it also differentiates it from them.}

\lm{Might help to nuance ``ARM boards, such as Raspberry Pi'' into something that directly motivates this choice. For example: you ported the software to a popular, off-the-shelf system that supports Linux, which can then help you say why these technical choices were made, i.e. because you wanted to do a practical evaluation of porting costs of legacy software.}

Software porting is costly and time consuming. As developers prefer to reuse
code instead of rewriting it, it is important to understand the framework for
quantifying the costs so that the process can be optimized based on these exact
costs. In this paper we port the Ixia network testing infrastructure on ARM
boards, such as Raspberry Pi, and extract the porting costs associated with this
process. By doing this we create a revised model of porting based on which we
extract the costs for our project. Furthermore we present the factors that
affect the costs and their direct correspondence with the porting model and
costs. Finally, we discuss the limitations of our porting model and make a
comparison with the old porting model.
}
